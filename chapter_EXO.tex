\chapter{Enriched Xenon Observatory}

The Enriched Xenon Observatory (EXO) is a set of two experiments, each a LXe time projection chamber (TPC) designed to study the double beta decay of the isotope \textsuperscript{136}Xe, and ultimately to search for the zero-neutrino mode.  There are several advantages to a LXe detector.  Xe scintillates at [around?] [xxx] nm, which is [efficiently collected by [type that the APDs are]] [reference]; so the Xe acts as a detection medium in addition to being the source of the double beta decay [reference? I didn't make up that kind of sentence].  Xe can also be continuously purified to maintain large electron lifetimes in the LXe.  Also, the ratio between observed scintillation light and remaining ionized electrons (drifted from the decay site by the TPC's electric field) exhibits a well-known microscopic anti-correlation [ref.], the understanding of which improves the energy resolution of the detector.  Finally, a LXe TPC approach offers the opportunity, [fairly] unique in double beta decay, to "tag" the daughter, in this case Ba, at the site of the double beta decay event (specifically, of course, the neutrinoless ones).

EXO-200 is the first of the two experiments, and has been operational since April of 20[xx](?).  It is a liquid xenon TPC designed to probe Majorana neutrino masses down to around 100~meV.  [EXO instrum. paper part I]  The following sections decribe the EXO-200 experiment, as well as nEXO, the next-generation tonne-scale liquid xenon TPC which is now in the design stages.  EXO-200 does not have Ba tagging implemented, but it is hoped that nEXO will.

\section{EXO-200}



\section{nEXO}