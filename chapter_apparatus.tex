\chapter{Apparatus}

{\color{gray}\emph{Should data results of diagnostic stuff, like Ba\textsuperscript{+} velocity (from pulses), be in this chapter?}}

This chapter describes the apparatus at Colorado State University, which we have used for all described studies of Ba fluorescence in SXe after deposition in vacuum.  Our main Ba source, the Ba\textsuperscript{+} ion source/beam, is first described, as well as the measurements (using Faraday cups) used for determining the number of ions we deposit.  A purely Ba neutral source is described.  The co-deposit of Ba/Ba\textsuperscript{+} with Xe gas onto a cold sapphire window, subsequent laser excitation, and finally the collection optics for the fluorescence, are described.

%\input{file path from here}
%\begin{equation}
%\label{name}, then refer to it with \ref{name}
%\overline{variable} for a bar over the variable

\section{Ion Beam}

\subsection{Barium Ion Source/Acceleration}

Barium ions are produced in a Colutron [type?] ion gun system [reference], as depicted in Fig. x.  A solid barium charge is placed into the hollowed end of a stainless steel rod, which is then inserted into the discharge chamber, near the hot filament.  The heated barium vaproizes, allowed to escape the hollowed rod around a loosely threaded set screw at the end of the rod.  The discharge chamber then fills with barium vapor.  A voltage is appled to the anode plate, which then creates a discharge, through the barium vapor, between the anode and the filament.  The resulting plasma, containing barium ions, then escapes through the small hole in the anode plate, where it enters the acceleration potential.  

The acceleration potential is 2~kV, between the ion source anode and an aperture, which constitutes the first element of the "acceleration lens" (Fig.~\ref{fig:testfig}).  The acceleration lens is an Einzel lens, the voltagesfor which are chosen to approximately collimate the ion beam for passage through the $E x B$ velocity filter.

\begin{figure}[H]
        \centering
                \includegraphics[width=.9\textwidth]{figures/figer.png}
                \caption{figyer}
\label{fig:testfig}
\end{figure}

\subsection{Velocity Filter, Lensing}

The $E \times B$ velocity filter selects $Ba^{+}$ by providing perpendicular electric and magnetic fields, which produce opposing forces on charged particles moving straight through the filter.  Those fields are chosen such that those forces are equal for $Ba^{+}$, according to Eqn. \ref{eqn:massfilt}:

\begin{equation}
\sigma = 1.
\label{eqn:massfilt}
\end{equation}

\noindent
Other ions will be deflected, while $Ba^{+}$ will continue along the beam path.

The full ion beam is shown in Fig. xxx.  The Decelerator lens can be used to reduce the beam energy, but is not needed for 2~keV beams, which are used in this work.  Einzel Lens 3 focuses the beam onto the main Faraday cup, which is used during experiments to meaure ion current.  The final set of deflection plates, H2 and V2, are also used during experiments to steer the beam for deposits.

\subsection{Ion Beam Pulsing}

To deposit small numbers of ions in a controlled manner, a set of pulsing plates can be used (Fig. xxxx).  When running in this mode, the pulsing plates are first placed at 200~V and -200~V to deflect the beam, and are pulsed to 0~V for 1~$\mu$s for each pulse.  

The pulses can be detected by the Induction Plates.  Since they use induction, they can be used to observe the pulses during an ion deposition (unlike using the Faraday cup to measure ion current during a DC deposit).  An example of an oscilloscope readout of the pulsing plate signa,l and subsequent induction plate signal, is shown in Fig. 5x.

\section{Ba Getter Source}

Ba "getters" are typically used in vacuum systems to improve vacuum by emitting Ba atoms, which grab gas molecules and hold them to the chamber walls.  We employ getters as a neutral Ba source in our system.

...

It is very helpful to have a completely different type of Ba source, to rule out any source-related quirks, e.g. source-produced impurities.

\section{Solid Xenon Matrix Deposition}

The final destination of the barium ions is in the solid xenon matrix, which is deposited onto a cold sapphire window.  Sapphire has good thermal conductivity, good optical transparency in the visible, and does not fluorescese in the wavelength region where barium fluoresces.  

Xenon freezes around 73~K (?) at our pressures ($0.5$ - $1 \times 10^{-7}$~Torr), so the window is cooled to temperatures below that.  The window is held to a cold finger (Fig. 6x, picture of), cooled by a -brand- cryostat. % -brand- said {\color{red}brand}, but that had errors

\subsection{Deposition Procedure}

Before barium ions are let through, xenon gas is allowed to flow, controlled by a leak valve, onto the cold sapphire window, where it freezes and begins growing the solid matrix.  The Faraday cup is then retracted, to clear the path for barium ions.  The cup serves as a shutter for DC deposition, or if pulsing is being used, they are performed at this time.  Barium ions land in the solid xenon as the matrix continues to grow.  The cup is then replaced, and the xenon leak stopped.