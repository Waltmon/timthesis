\documentclass[PhD, copyrightpage]{csuthesis} %draft means that it won't actually render any pictures (graphs) it will just include space for them.
% As soon as you get this to compile, your next step
% should be to remove the draft option, so that your
% figures will show up.  Right now it's there so that
% the document will compile as is, with sample figures.

% If the grad school gives you grief for the smallcaps
% (all caps but capital capitals slightly larger) then
% you can use the "nosmallcaps" option to eliminate
% all smallcaps from the preliminary pages.

% The "copyrightpage" option tells the document class
% to generate a copyright page.  If you don't want a
% copyright page, erase that option.

% Now, change the following fields to match your thesis:
\title{Barium Tagging in Solid Xenon for the nEXO Neutrinoless Double Beta Decay Experiment}
\author{Timothy Walton}
\departmentname{Department of Physics}
\gradterm{Fall} %The semester that you are going to turn this in.
% The year will automatically be set to the current year.  If you
% want to modify the year manually, use this command:
%\gradyear{2014}
% But otherwise, ignore that command.
\advisor{William M. Fairbank, Jr.}
%\coadvisor{Test Test} %optional
\committee{Robert Wilson \and someone \and Alan Van Orden} %separate committee names with \and
%committee names should not have any honorifics (i.e. NO Dr., PhD, professor, etc.)  Just names.

% you do not need these newcommand lines.
% They are my personal shorthand for derivatives.
\newcommand{\deriv}[2]{\frac{\mathrm{d}#1}{\mathrm{d}#2}}
\newcommand{\dby}[1]{\frac{\mathrm{d}}{\mathrm{d}#1}}
\newcommand{\pd}[2]{\frac{\partial#1}{\partial#2}}
\newcommand{\pby}[1]{\frac{\partial}{\partial#1}}

\usepackage{lipsum}%this package provides nonsense text for testing document layouts.  Not needed for real thesis.
\usepackage{color}
\usepackage{xcolor}
\usepackage{amsmath}

% a couple useful options that you may wish to uncomment:
% \setcounter{tocdepth}{3} %more depth in table of contents
% \numberwithin{equation}{chapter} %display eqn # as 1.3, not 3
% \renewcommand{\bibname}{References} %change the name of your bibliography
% numberwithin commands for table, figure, and section are 
% already included in the .cls file.  You can search for 
% them and remove them if you want to.

\begin{document}
\frontmatter
%turns the page numbering to roman, and does some other stuff.

\begin{abstract}

\begin{center}\textsc{Imaging Single Barium Atoms in Solid Xenon for Barium Tagging in the nEXO Neutrinoless Double Beta Decay Experiment}\end{center}

\vspace{6mm}

The nEXO experiment will search for neutrinoless double beta decay of the isotope \textsuperscript{136}Xe in a ton-scale liquid xenon time projection chamber, in order to probe the Majorana nature of neutrinos. Detecting the daughter \textsuperscript{136}Ba of double beta decay events, called barium tagging, is a technique under investigation which would provide a veto for a background-free measurement. This would involve detecting a single barium ion from within a macroscopic volume of liquid xenon. One proposed barium tagging method is to trap the barium ion in solid xenon at the end of a cold probe, and then detect it by its fluorescence in the solid xenon.  In this thesis, new studies on the spectroscopy of deposits of Ba and Ba\textsuperscript{+} in solid xenon are presented.  Imaging of barium atoms in solid xenon is demonstrated with sensitivity down to the single atom level.  Achievement of this level of sensitivity is a major step toward barium tagging by this method.

%in the search for this decay
\end{abstract}
%\begin{abstract}
%abztrakt
%\end{abstract}
% note that for amsbook (which is basically the class we are copying), the abstract must be declared before the title.  It prints as part of maketitle.
% personally, I put my abstract definition in a separate file, abs_for_diss.tex, and used \input{abs_for_diss} here.
% The grad school requires an abstract.

\begin{acknowledgements}
Thanks of course goes to my advisor Bill Fairbank 
\end{acknowledgements}
%this is optional.  It must be declared before maketitle, a lot like abstract.  Consider having this in a separate file, like ack.tex, and then using \input{ack}

\maketitle
% maketitle is a huge command in a small package.  It will render the title page, copyright page, acknowledgements, abstract, and probably a bunch of other stuff that I'm forgetting about.  If you choose not to define copyright, acknowledgements, etc, it automatically doesn't render them.

% other optional frontmatter could go somewhere in here.  dedication, biography, etc.  Grad school lists approved frontmatter pages somewhere.
% most frontmatter than I consider silly is not yet automated.  Also most frontmatter doesn't have any explicit style requirements, so I feel justified in ignoring it.

\tableofcontents %don't move this around too much
%\listoftables %optional, but this is the spot it should go
%\listoffigures %optional, but this is the right location for it

\mainmatter %switches page numbering to normal, resets page counter, etc.
% This is the start of the main writing, but there are a couple more
% notes that I think might be helpful, tossed in here and there with
% the format testing stuff.

\chapter{Introduction}

Neutrinos have provided illumination as well as great challenge to physics since their discovery.  They first entered our consciousness through W. Pauli, who proposed in 1930 the existence of a neutral, unobserved particle to explain the apparent violation of energy conservation in beta decay. He admitted that neutrinos (then deemed ``neutrons'' -- what we now know as neutrons had not been discovered yet either) should be difficult to observe experimentally, but also that it seemed unlikely that they would never have been noticed before [ref: Pauli letter, or a following paper?].  As it turns out, they are much more difficult to observe than he predicted; they aren't likely to be noticed without extreme experimental techniques.

A theory formulated in 1933 by E. Fermi for beta decay, including the neutrino, would be the beginnings of weak theory, and the development of the very successful Standard Model of particle physics.  But neutrinos continued to challenge theory with the discovery of non-zero neutrino mass, and they remain at the forefront of our exploration of the universe.

The possibility that neutrinos are Majorana particles makes the search for neutrinoless double beta decay very important for the further development of particle theory.  Majorana formulation can describe the origin of neutrino mass, and possibly explain why the mass is very small via the Seesaw Mechanism [ref].  Observation of neutrinoless double beta decay would simultaneously demonstrate that neutrinos are Majorana particles, as well as give a measurement of the absolute mass itself [ref? this is said later too].

To motivate barium tagging, this chapter outlines the current theory for neutrinos, and then describes the neutrinoless double beta decay experiments EXO-200 and nEXO.

\section{Neutrinos}

Neutrinos are chargeless leptons which only interact via the weak force (and gravity).  There are three known ``flavors'' of neutrinos, each corresponding to one of the three known leptons:  $\nu_{e}$, $\nu_{\mu}$, and $\nu_{\tau}$.  These are the eigenstates in the basis of the weak force, so they are the states in which a neutrino will interact via the weak force.

\subsection{Neutrino Oscillation and Mass}

{\color{red}Neutrinos exhibit mixing between their energy eigenstates and their weak force eigenstates, and these are not the same basis. }  {\color{red}\textbf{not really:}  This means that a flavor eigenstate is not a stationary state} {\color{gray}[the fact is that quark mixing happens, and one theory was that neutrinos would do something similar, and then oscillation is discovered...etc...um...well, they could be a different basis, but apparently they would not oscillate if their masses were all zero. You might want to work through the equations assuming zero mass and see what time-evolving a flavor state does ... could they still have 3 different energy states (i.e., momenta)?]--- }a neutrino which begins as a pure flavor state (as all neutrinos will, coming out of a quantum process involving one of the three leptons) will oscillate into the other two flavors as it evolves in time, i.e. the probability of measuring it to be one of the other two flavors is no longer zero.

{\color{gray}\emph{History:  }The first indications of neutrino oscillation came around 1970 with the Ray Davis Experiment [ref.], which measured the flux (at Earth) of solar electron neutrinos.  The flux measured was quite a bit lower than predicted by solar models, and this became known as the Solar Neutrino Problem.  The discovery of neutrino oscillation in the late 1990s [ref.] solved this problem, as only a fraction of the sun's neutrinos, produced as pure electron neutrinos, would interact as such.}

The very small mass of a neutrino, specifically relative to its momentum, lets one write its Hamiltonian in terms of mass squared differences $\Delta m_{ij}^{2} = m_{i}^{2} - m_{j}^{2}$, where $i$,$j$ = 1,2,3, referring to what we then call mass states.  The mass basis is really the energy basis with the small mass approximation, along with dropping some constant terms in the Hamiltonian (which do not affect time evolution).  Writing the time evolution in terms of mass squared differences means that neutrino oscillation experiments can produce measurements of these differences.  In fact, the discovery of neutrino oscillation was the first (and only, so far) demonstration that neutrinos have a non-zero mass.  Without neutrino mass (particularly without differences between the masses of the mass states), neutrinos would not oscillate.

Neutrino oscillation experiments also provide measurements on the amount of mixing between the flavor basis and the mass basis.  We define the mixing between them by a rotation in terms of three mixing angles, $\theta_{12}$, $\theta_{23}$, and $\theta_{13}$.  Transformation between the flavor and mass bases is done with the following unitary matrix, called the Pontecorvo--Maki-–Nakagawa–-Sakata (PMNS) matrix:

\begin{equation}
\begin{aligned}
U &= \begin{pmatrix}
1 & 0 & 0 \\
0 & c_{23} & s_{23} \\
0 & -s_{23} & c_{23} \end{pmatrix}
\begin{pmatrix}
c_{13} & 0 & s_{13} e^{-i \delta} \\
0 & 1 & 0 \\
-s_{13} e^{i \delta} & 0 & c_{13} \end{pmatrix}
\begin{pmatrix}
c_{12} & s_{12} & 0 \\
-s_{12} & c_{12} & 0 \\
0 & 0 & 1 \end{pmatrix}
\begin{pmatrix}
1 & 0 & 0 \\
0 & e^{i \alpha_{1}/2} & 0 \\
0 & 0 & e^{i \alpha_{2}/2} \end{pmatrix} \\
& = \begin{pmatrix}
c_{12} c_{13} & s_{12} c_{13} & s_{13} e^{-i \delta} \\
-s_{12} c_{23} - c_{12} s_{23} s_{13} e^{i \delta} & c_{12} c_{23} - s_{12} s_{23} s_{13} e^{i \delta} & s_{23} c_{13} \\
s_{12} s_{23} - c_{12} c_{23} s_{13} e^{i \delta} & -c_{12} s_{23} - s_{12} c_{23} s_{13} e^{i \delta} & c_{23} c_{13} \end{pmatrix}
\begin{pmatrix}
1 & 0 & 0 \\
0 & e^{i \alpha_{1}/2} & 0 \\
0 & 0 & e^{i \alpha_{2}/2} \end{pmatrix}
\end{aligned}
\label{eqn:umatrix}
\end{equation}

\noindent
where $c_{ij} = \cos \theta_{ij}$ and $s_{ij} = \sin \theta_{ij}$.  $\delta$ is a phase factor related to lepton CP violation, and $\alpha_{i}$ are Majorana phases.

Studying oscillations of neutrinos from different kinds of sources, with different energies and path lengths, can isolate sensitivities to the different parameters {\color{gray}(not really sure if this is the right thing to say)}.  For example, the study of solar neutrinos (neutrinos emanating from nuclear fusion reactions in the core of the sun) provides sensitivity to $\theta_{12}$ and $\Delta m_{12}^{2}$ {\color{gray}(\emph{right? $\theta_{12}$ may not be specifically solar...})}.  Beamline neutrino detectors can be designed for maximum sensitivity to parameters.  The parameters so far measured are as follows in Table  \ref{table:nu_osc_vals}:

\begin{table}[!htbp]
\caption{up to date values with references, and denote ``solar'', ``atmos.'', etc.} %not sure what [Small Table], between \caption and {}, w/ no spaces, does
\label{table:nu_osc_vals}
\begin{tabular}{c|c}
Parameter & Measurement \\
\hline
$\Delta m_{12}^{2}$ & \\
$|\Delta m_{31}^{2}|$ & \\
$\sin^{2} \theta_{12}$ & \\
$\sin^{2} \theta_{23}$ & \\
$\sin^{2} \theta_{13}$ & \\
\end{tabular}
\end{table}

Note that only the absolute value of $\Delta m_{31}^{2}$ is known.  As a consequence, there are two possibilities for the hierarchy of the three neutrino masses.  These are called the Normal and Inverted Hierarchies, as shown in Fig. \ref{fig:numasshier}.

\begin{figure}[H]
        \centering
                \includegraphics[width=.5\textwidth]{figures/hierarchy_alterred.png}
                \caption{The two possible hierarchies of neutrino masses.  The colors depict the mixing between the mass and flavor bases. {\color{red}\textbf{[ref]}}}
\label{fig:numasshier}
\end{figure}

\noindent
The correct mass hierarchy remains unknown, but next-generation neutrino experiments, possibly including nEXO, will be able to discern this.

Neutrino oscillation demonstrates that neutrinos have non-zero mass, and though oscillation experiments can measure the mass squared differences, we still do not have a measurement of the absolute masses of the three neutrinos.  {\color{gray}\emph{History:  }mention Pauli's first statement of limit near electron mass?  Then maybe say the 0 assumption, which I think came from Fermi's beta decay theory, but if you mention that above, just refer to it here.}

The current upper limits on the mass come from...(KATRIN(\rightarrow\nu_{e})?  Cosmology(\rightarrow sum)?

Neutrinoless double beta decay experiments like EXO-200 can put upper limits on specifically the Majorana neutrino mass (i.e., upper limits on the neutrino mass if neutrinos are indeed Majorana particles).  As discussed in the next chapter, [EXO-200 and KamLAND (sp?) ZEN (sp?) together provide the strongest Majorana neutrino mass upper limit of [] (IS THAT TRUE?)].

Neutrino oscillation and non-zero neutrino mass are physics beyond the Standard Model (SM) of particle physics, and though much has been discovered through oscillation experiments, there is much yet to learn about neutrinos. Since they are chargeless, they may be Majorana particles, and their small mass could be explained by the See-saw Mechanism [ref]. Majorana particles are their own anti-particle, and this, along with the discovery that neutrinos have mass, allows for a unique test of the Majorana (vs. Dirac) nature of neutrinos: neutrinoless double beta decay.

\subsection{Neutrinoless Double Beta Decay}

Double beta decay is the simultaneous emission of two electrons from a nucleus.  Two-neutrino double beta decay, shown in Fig. \ref{fig:feynman_diags}(left), is allowed by the Standard Model and has been observed in several isotopes which are listed in Table (table).  Similar to beta decay, a neutrino accompanies each electron in this decay, broadening the spectrum of the summed electron energy. This is a second-order process, making it a rare decay, and requiring low backgrounds to measure.

\begin{figure}[H]
        %\centering
        \begin{subfigure}
                \includegraphics[width=.35\textwidth]{figures/feynman_2nu_quarks.png}
                %\caption{barf}
%        %\end{subfigure}
        %\begin{subfigure}
                \includegraphics[width=.35\textwidth]{figures/feynman_0nu_quarks.png}
                \caption{Two-neutrino (left) and Neutrinoless (right) double beta decay.}
        \end{subfigure}
        \label{fig:feynman_diags}
\end{figure}

Neutrinoless double beta decay, shown in Fig. \ref{fig:feynman_diags}(right), is a postulated mode of double beta decay. In this case, the neutrino is exchanged as a virtual particle (which would require that it is a Majorana particle), and there are no neutrinos in the final products. If discovered, not only would neutrinos be determined Majorana particles, but their absolute mass could also be measured in the form of an effective electron neutrino mass, since the rate of neutrinoless double beta decay will depend on the absolute neutrino mass as shown in Eqn. \ref{eqn:rate_vs_mass}:

\begin{equation}
T_{1/2}^{0\nu} = (G^{0\nu}(Q,Z)|M^{0\nu}|^{2}\braket{m_{\nu}}^{2})^{-1}
\label{eqn:rate_vs_mass}
\end{equation}

\noindent
where $T_{1/2}^{0\nu}$ is the $0\nu\beta\beta$ half-life,  $G^{0\nu}$ is a known phase space factor, and $M^{0\nu}$ is a model-dependent nuclear matrix element. The effective electron neutrino mass $\braket{m_{\nu}}$ is the expectation value of the mass for a pure electron neutrino:

\begin{equation}
\braket{m_{\nu}} = \sum\limits_{i} U_{ei}^{2} m_{i}.
\label{eqn:effectivemass}
\end{equation}

The sum of the energies of the emitted electrons in double beta decay will serve as the distinction between the two-neutrino and zero-neutrino modes, shown in Fig. \ref{fig:spectrum_bb}. In the two-neutrino mode, the total decay energy is shared probabilistically between the electrons and the neutrinos (the nucleus recoil energy is negligible), resulting in a broad distribution in the summed electron energy. (Recall the similarly broad electron energy in single beta decay, which ultimately led to discovery of the neutrino involved.) But in the zero-neutrino mode, all of the decay energy is carried away by the two electrons, resulting in only a single allowed value for the summed electron energy -- a peak in the summed electron energy spectrum at the Q-value. 

\begin{figure}[H]
        \centering
                \includegraphics[width=.7\textwidth]{figures/spectrum_bb.png}
                \caption{Conceptual two-neutrino (blue) and zero-neutrino (red) double beta decay spectra.}
\label{fig:spectrum_bb}
\end{figure}

The rarity of double beta decay (see the very long half lives in Table (table)) requires very low backgrounds, especially around the Q-value for the $0\nu\beta\beta$ search. The next sections describe EXO-200 and it's next-generation successor, nEXO.

\section{Enriched Xenon Observatory}

The Enriched Xenon Observatory (EXO) is a set of two experiments, each a LXe time projection chamber (TPC) designed to study the double beta decay of the isotope \textsuperscript{136}Xe, and ultimately to search for the zero-neutrino mode.  There are several advantages to a LXe detector.  Xe is extremely transparent, and scintillates at [around?] [xxx] nm, which is [efficiently collected by [type that the APDs are]] [reference]; so the Xe acts as a detection medium in addition to being the source of the double beta decay [reference? I didn't make up that kind of sentence].  Xe can be continuously purified to maintain large electron lifetimes in the LXe.  Also, the ratio between observed scintillation light and remaining ionized electrons (drifted from the decay site by the TPC's electric field) exhibits a well-known microscopic anti-correlation [ref.], the understanding of which improves the energy resolution of the detector.  Finally, a LXe TPC approach offers the opportunity, [fairly] unique in double beta decay, to reach in and identify, or ``tag'', the daughter Ba\textsuperscript{++} at the site of the double beta decay event, which would provide a background-free identification of neutrinoless double beta decay. Barium tagging is the focus of our group at CSU and is the subject of this thesis.

EXO-200 is the first of the two experiments, and has been operational since April of 20[xx](?).  It is a liquid xenon TPC designed to probe Majorana neutrino masses down to around 100~meV.  [EXO instrum. paper part I]  The following sections decribe the EXO-200 experiment, as well as nEXO, the next-generation tonne-scale liquid xenon TPC which is now in the design stages.  EXO-200 does not have barium tagging implemented, but it is hoped that nEXO will. 

\subsection{EXO-200}



\subsection{nEXO}

\noindent
{\color{gray}\textbf{old stupid intro:}

The study of the neutrino has required extremes in experimental technique from the beginning.  Neutrinos were described by W. Pauli, who first proposed their existence to remedy an apparent violation of energy conservation in beta decay, as being [impossible to detect] [ref.].  Rather, it requires a great deal of sensitiviy, ingenuity, and hardship (just "ingenuity and hardship"?  sensitivity may be redundant) to observe them, and it was [] years before they were first observed by [Reines and Cowan] in [], by [] [ref.].  (is "rather, ..." too demoting-sounding?  it is absolutely not meant to be, of course)

Neutrino experiments of greater discovery power have been developed around the world, and command large collaborations of scientists.  ummmmmmmm  this is supposed to kind of allude to barium tagging as an extreme technique

Neutrinoless Double Beta Decay experiments like EXO are a different kind of neutrino experiment, not detecting neutrinos directly, but searching for an effect (neutrinoless double beta decay itself) which would demonstrate the Majorana nature of neutrinos.  A liquid xenon experiment like EXO provides a the challenging opportuniy for another extreme experimental technique, barium tagging, where a single barium ion would be observed at a specific double beta decay site in the volume.  This thesis is part of an exploration of one promising barium tagging technique.  (these things may be saved for the EXO chapter... idk).

From the first formulation of beta decay theory by E. Fermi [ref.], neutrinos have provided an avenue into a world of new physics, and they continue to be such an avenue.  Questions which may be answered by this up and coming generation of neutrino experiments are expected to help explain how the universe came to be this way.

[lead into barium tagging discussion]

\section{something like "Can We Do This?", but probably not that at all.}

A liquid xenon double beta decay detector allows unique access to the daughters of decays in the liquid volume.  The feasibility of grabbing and detecting a single ion from the volume is what must be determined next.}
\chapter{Theory}

Theory relevant to the spectroscopy and detection of single Ba/Ba\textsuperscript{+} in SXe matrices is discussed.  

\section{Ba/Ba\textsuperscript{+} Spectroscopy in Vacuum}

The lowest-lying energy levels in vacuum for Ba and Ba\textsuperscript{+} are shown in Fig. \ref{fig:elevs}.  For Ba, the main transition is between the ground $6s^{2}$ $^{1}$S$_{0}$ to the excited $6s6p$ $^{1}$P$_{1}$ state.  Spin{\color{red}(?)}-suppressed transitions between the P state and three metastable D states results in a decay in to a D state after about 350 excitations.  For Ba\textsuperscript{+}, two strong transitions exist between the ground $6s$ $^{2}$S$_{1/2}$ \textbf{{\color{red}is the doublet correct??}} and the $6p$ $^{2}$P$_{1/2}$ and $6p$ $^{2}$P$_{3/2}$ excited states.  Transitions to the two metastable D states are higher than for the atom, resulting in a decay into a D state after about 4 excitations.

\begin{figure}[H]
	\includegraphics[width=.35\textwidth]{figures/BaLevs_atom.pdf}
	\includegraphics[width=.35\textwidth]{figures/BaLevs_ion.pdf}
	\caption{Energy level diagrams for (a) Ba and (b) Ba\textsuperscript{+} {\color{red}\textbf{Is the 2 correct in 2S1/2 in the ion?}  Get rid of high P states in Ba}}
    \label{fig:elevs}
\end{figure}

These energy levels and their transition rates are well known, and are documented in the NIST tables [ref].  Single atom/ion detection by spectroscopy generally requires lasers to provide transitions out of the metastable D states once the atom/ion decays into one of them, in addition to the main excitation laser.  For the atom, this requires three additional infrared lasers, and single atom trapping/detection in a magneto-optical trap (MOT) is achieved in [ref Ba MOT].  Single Ba\textsuperscript{+} observation requires only two lasers if the $^{2}$P$_{1/2}$ excited state is used.  This is demonstrated in [ref something that does this ... is there something?  How about the Carleton group?].

\section{Matrix Isolation Spectroscopy}

The spectroscopy of a species trapped in an inert solid matrix is called matrix isolation spectroscopy, the concept of which was pioneered around 19?? by [that one guy] [ref ... maybe [1] of ba spec].  Though absorption and emission are significantly broadened and shifted, a species can retain similarities to its vacuum counterpart, such as quantum numbers.

Studies of some matrix isolation systems have been made.  The most thoroughly studied has been Na(?) in solid matrices of (Ar, Kr, Xe ?)... (Na? Mg?)...[refs]

We are of course interested in the spectroscopy of Ba and/or Ba\textsuperscript{+} in SXe matrices, and this particular system has not been studied until recently.  The first report of the spectroscopy of neutral Ba in SXe, along with candidate fluorescence peaks for Ba\textsuperscript{+} in SXe, was published by our group in [ref ba spec], and conversations following this publication with ???'s group in ??? have confirmed our basic observations of the absorption spectrum of Ba in SXe.  

From here forward I will refer to the host species as Xe and the guest as Ba/Ba\textsuperscript{+}, specific to our system.  The leading interaction between {\color{gray}[put this here if the van der waals is only necessarily the force for Ba in particular]} the Ba/Ba\textsuperscript{+} and a neighboring Xe atom is a {\color{gray}(the)} Van Der Waals (sp?) force, an induced dipole-dipole interaction.  Xe is the most polarizable (sp?) of the noble gases.  

The forms for this force(?) is shown in Eqn. [ref van der ba] for Ba and Eqn. [ref van der ba+] for Ba\textsuperscript{+}.  {\color{gray}Talk about it.}

This binding between the Ba/Ba\textsuperscript{+} and its surrounding Xe atoms results in vibrational modes, and the Franck Condon (sp?) principle applies.  Fig. [ref fig franckcondon] helps illustrate this effect.  In a cold matrix, the system will be in the ground vibrational state before excitation.  The distribution of the wavefunction, even for the ground state alone, overlaps in space in general with more than one of the excited state vibrational modes, resulting in a broadening in energy of the absorption.

Rapid decay occurs to the lowest vibrational mode in the excited state before electronic decay can occur [ref] {\color{red}[is this just for Ba?]}.  Then a similar broadening in emission energy occurs as several overlapping vibrational mode wavefunctions exist in the ground electronic state, and a redshift is also observed as some energy was dissipated into phonons in the crystal in vibrational mode decays.  {\color{blue}\textbf{(How can you get a blueshift?)}}

Shifts and broadening in absorption and emission depend in general on the distribution of Xe atoms around the Ba/Ba\textsuperscript{+}.  The fcc crystal structure of the Xe restricts these environments to discrete number of so-called matrix sites, defined by (a) whether the Ba/Ba\textsuperscript{+} is {\color{red}[intersituational or whatever -- is that still a thing?]}, and (b) the number of vacancies in the Xe matrix surrounding the Ba/Ba\textsuperscript{+}.  Experimental observation of emission peaks from different matrix sites of [Na] in solid noble gases is reported in [ref(s)], with theoretical calculations in [ref(s)] attributing observed peaks to specific vacancy distributions defining the matrix sites (maybe this has been done?).

Energy level transition probabilities can also be affected in a matrix.  Distortions in electron wavefunction shapes by asymmetric matrix sites can affect radiative transitions by altering parity [ref].  If electronic potential energy curves cross each other, nonradiative transitions can also become allowed for otherwise forbidden transitions [ref].  {\color{red}(Is a phonon emission called a nonradiative transition / does this actually happen?)}

The detectability of a single atom/ion can depend greatly on altered transition rates.  For example, in the Ba atom, matrix-allowed decay of the $^{1}$P$_{1}$ to the $^{3}$P states could be much stronger than the main transition back to ground, which would suppress fluorescence and make single-atom detection impossible.  But the matrix could also help by strengthening decays from the D states back to ground, eliminating any need for re-pump lasers to keep the transition cycle going.  
\chapter{Enriched Xenon Observatory}

The Enriched Xenon Observatory (EXO) is a set of two experiments, each a LXe time projection chamber (TPC) designed to study the double beta decay of the isotope \textsuperscript{136}Xe, and ultimately to search for the zero-neutrino mode.  There are several advantages to a LXe detector.  Xe scintillates at [around?] [xxx] nm, which is [efficiently collected by [type that the APDs are]] [reference]; so the Xe acts as a detection medium in addition to being the source of the double beta decay [reference? I didn't make up that kind of sentence].  Xe can also be continuously purified to maintain large electron lifetimes in the LXe.  Also, the ratio between observed scintillation light and remaining ionized electrons (drifted from the decay site by the TPC's electric field) exhibits a well-known microscopic anti-correlation [ref.], the understanding of which improves the energy resolution of the detector.  Finally, a LXe TPC approach offers the opportunity, [fairly] unique in double beta decay, to "tag" the daughter, in this case Ba, at the site of the double beta decay event (specifically, of course, the neutrinoless ones).

EXO-200 is the first of the two experiments, and has been operational since April of 20[xx](?).  It is a liquid xenon TPC designed to probe Majorana neutrino masses down to around 100~meV.  [EXO instrum. paper part I]  The following sections decribe the EXO-200 experiment, as well as nEXO, the next-generation tonne-scale liquid xenon TPC which is now in the design stages.  EXO-200 does not have Ba tagging implemented, but it is hoped that nEXO will.

\section{EXO-200}



\section{nEXO}
\chapter{Apparatus}

{\color{gray}\emph{Should data results of diagnostic stuff, like Ba\textsuperscript{+} velocity (from pulses), be in this chapter?}}

This chapter describes the apparatus at Colorado State University, which we have used for all described studies of Ba fluorescence in SXe after deposition in vacuum.  Our main Ba source, the Ba\textsuperscript{+} ion source/beam, is first described, as well as the measurements (using Faraday cups) used for determining the number of ions we deposit.  A purely Ba neutral source is described.  The co-deposit of Ba/Ba\textsuperscript{+} with Xe gas onto a cold sapphire window, subsequent laser excitation, and finally the collection optics for the fluorescence, are described.

%\input{file path from here}
%\begin{equation}
%\label{name}, then refer to it with \ref{name}
%\overline{variable} for a bar over the variable

\section{Ion Beam}

\subsection{Barium Ion Source/Acceleration}

Barium ions are produced in a Colutron [type?] ion gun system [reference], as depicted in Fig. x.  A solid barium charge is placed into the hollowed end of a stainless steel rod, which is then inserted into the discharge chamber, near the hot filament.  The heated barium vaproizes, allowed to escape the hollowed rod around a loosely threaded set screw at the end of the rod.  The discharge chamber then fills with barium vapor.  A voltage is appled to the anode plate, which then creates a discharge, through the barium vapor, between the anode and the filament.  The resulting plasma, containing barium ions, then escapes through the small hole in the anode plate, where it enters the acceleration potential.  

The acceleration potential is 2~kV, between the ion source anode and an aperture, which constitutes the first element of the "acceleration lens" (Fig.~\ref{fig:testfig}).  The acceleration lens is an Einzel lens, the voltagesfor which are chosen to approximately collimate the ion beam for passage through the $E x B$ velocity filter.

\begin{figure}[H]
        \centering
                \includegraphics[width=.9\textwidth]{figures/figer.png}
                \caption{figyer}
\label{fig:testfig}
\end{figure}

\subsection{Velocity Filter, Lensing}

The $E \times B$ velocity filter selects $Ba^{+}$ by providing perpendicular electric and magnetic fields, which produce opposing forces on charged particles moving straight through the filter.  Those fields are chosen such that those forces are equal for $Ba^{+}$, according to Eqn. \ref{eqn:massfilt}:

\begin{equation}
\sigma = 1.
\label{eqn:massfilt}
\end{equation}

\noindent
Other ions will be deflected, while $Ba^{+}$ will continue along the beam path.

The full ion beam is shown in Fig. xxx.  The Decelerator lens can be used to reduce the beam energy, but is not needed for 2~keV beams, which are used in this work.  Einzel Lens 3 focuses the beam onto the main Faraday cup, which is used during experiments to meaure ion current.  The final set of deflection plates, H2 and V2, are also used during experiments to steer the beam for deposits.

\subsection{Ion Beam Pulsing}

To deposit small numbers of ions in a controlled manner, a set of pulsing plates can be used (Fig. xxxx).  When running in this mode, the pulsing plates are first placed at 200~V and -200~V to deflect the beam, and are pulsed to 0~V for 1~$\mu$s for each pulse.  

The pulses can be detected by the Induction Plates.  Since they use induction, they can be used to observe the pulses during an ion deposition (unlike using the Faraday cup to measure ion current during a DC deposit).  An example of an oscilloscope readout of the pulsing plate signa,l and subsequent induction plate signal, is shown in Fig. 5x.

\section{Ba Getter Source}

Ba "getters" are typically used in vacuum systems to improve vacuum by emitting Ba atoms, which grab gas molecules and hold them to the chamber walls.  We employ getters as a neutral Ba source in our system.

...

It is very helpful to have a completely different type of Ba source, to rule out any source-related quirks, e.g. source-produced impurities.

\section{Solid Xenon Matrix Deposition}

The final destination of the barium ions is in the solid xenon matrix, which is deposited onto a cold sapphire window.  Sapphire has good thermal conductivity, good optical transparency in the visible, and does not fluorescese in the wavelength region where barium fluoresces.  

Xenon freezes around 73~K (?) at our pressures ($0.5$ - $1 \times 10^{-7}$~Torr), so the window is cooled to temperatures below that.  The window is held to a cold finger (Fig. 6x, picture of), cooled by a -brand- cryostat. % -brand- said {\color{red}brand}, but that had errors

\subsection{Deposition Procedure}

Before barium ions are let through, xenon gas is allowed to flow, controlled by a leak valve, onto the cold sapphire window, where it freezes and begins growing the solid matrix.  The Faraday cup is then retracted, to clear the path for barium ions.  The cup serves as a shutter for DC deposition, or if pulsing is being used, they are performed at this time.  Barium ions land in the solid xenon as the matrix continues to grow.  The cup is then replaced, and the xenon leak stopped.
\chapter{Results}

First I describe the emission spectra for neutral Ba in SXe, which are first reported in the theses of B. Mong [ref] and S. Cook [ref].  Improved studies of several emission peaks are reported.  The bleaching of several of these peaks is carefully observed [{\color{do a correction on p-meter sensitive area and on p-meter quantum efficiency}}], and a model of atomic transition rates is fit to this data.  Improved excitation spectra are achieved for all observed Ba emission peaks.  Images of $\leq$ 10\textsuperscript{3\textbf{?}} Ba atoms are achieved using the bleaching peaks.  Ultimately, images of Ba atoms at the few-atom level are achieved using the 619-nm fluorescence peak.  Improved studies of candidate fluorescence peaks of Ba\textsuperscript{+} in SXe are also reported.  

\section{um}
\chapter{Conclusions}

The achievement of single-atom-level imaging of Ba atoms in SXe is significant progress toward the ability to tag barium daughters in the nEXO LXe TPC.  Improved studies of the spectroscopy of Ba in SXe have been presented and utilized in reaching this level of sensitivity with the 619-nm fluorescence, as well as in imaging the 577- and 591-nm fluorescence for deposits down to $\leq$ around $10^{3}$ atoms.  Blue excitation of Ba\textsuperscript{+} deposits in SXe has revealed several newly reported emission peaks which are candidates for matrix-isolated Ba\textsuperscript{+} ions.

%neutralized Ba\textsuperscript{+} deposits in SXe have been presented.  Of particular interest was temperature and annealing effects, as well as bleaching effects of the 577-, 591-, and 619-nm fluorescence peaks.  Fluorescence was maximal in deposits made around 50~K and observed at 11~K.  The 577- and 591-nm peaks experience similarly rapid bleaching.  A model applied to bleaching data of the 591-nm peak demonstrated agreement with optical pumping rates of Ba in vacuum near the beginning of the bleaching process, with subsequent deviation.  The 619-nm peak experiences much lower bleaching rates.

%Understanding of these effects aided in the design of imaging experiments.  Imaging of Ba atoms in SXe in a focused laser region was demonstrated down to the $\leq 10^{3}$-atom level using the 577- and 591-nm peaks, and down to the single-atom level using the low-bleaching 619-nm peak.

\section{Future Work}
\label{sec:future}

The next major demonstration will be imaging separate Ba atoms in SXe in scanned images.  As indicated in Sec. \ref{sec:scanning}, overcoming the challenge of variable surface background emission is the first priority.  Studies of fluorescence lifetimes of the 619-nm signal vs. the background using a pulsed laser and a photo-multiplier tube for fast detection will be explored, in order to determine whether a photon-sorting technique may remove background.  The atom counting inherent in scanned images will provide the first measurement of the percentage of deposited Ba\textsuperscript{+} ions visible via the 619-nm fluorescence.

Exploration of re-pumping wavelengths for the 577- and 591-nm peaks, as well as the 570- and 601-nm peaks, will be done using tunable lasers in the infrared.  The ability to count atoms emitting with these different wavelengths would be very valuable.  It is not yet known how deposits made by freezing on a probe in LXe will look.  The ability to count ions will be pursued by continuing study of the candidate Ba\textsuperscript{+} emission peaks with blue excitation.

Ultimate demonstrations of Ba tagging in SXe will be done by grabbing out of LXe on cold probes.  Construction of an apparatus is occurring simultaneously in Fairbank's laboratory by Adam Craycraft, in which a cold probe can be lowered into a LXe cell.  Ba\textsuperscript{+} will be produced by laser ablation of Ba metal and drifted into the LXe and onto the cold probe by electrodes.  In order to reach temperatures around 10~K, where the Ba fluorescence is strong, the probe will be raised to an isolated pumping region for further cooling and observation.

%pics?

%\backmatter %...this doesn't seem to do anything

\appendix %this switches to apppendix mode.  Now any new chapters will be appendices instead of chapters.

\chapter{Supplementary Material} %this will be displayed as an appendix, not a chapter.  Appendices are at the same level in the hierarchy as chapters.

make these also into separate files plz

\section{Some Sample Material}

Appendix is a strange name.  Did the name for the written material come before the name of the organ? Here~\cite{infodim} is a citation in an appendix.

\chapter{Another Supplement}

% still missing some, like backmatter (what is that?), and bibliography:

% <same preamble as we have here>
% \begin{document}
% \frontmatter
% \begin{abstract}

\begin{center}\textsc{Imaging Single Barium Atoms in Solid Xenon for Barium Tagging in the nEXO Neutrinoless Double Beta Decay Experiment}\end{center}

\vspace{6mm}

The nEXO experiment will search for neutrinoless double beta decay of the isotope \textsuperscript{136}Xe in a ton-scale liquid xenon time projection chamber, in order to probe the Majorana nature of neutrinos. Detecting the daughter \textsuperscript{136}Ba of double beta decay events, called barium tagging, is a technique under investigation which would provide a veto for a background-free measurement. This would involve detecting a single barium ion from within a macroscopic volume of liquid xenon. One proposed barium tagging method is to trap the barium ion in solid xenon at the end of a cold probe, and then detect it by its fluorescence in the solid xenon.  In this thesis, new studies on the spectroscopy of deposits of Ba and Ba\textsuperscript{+} in solid xenon are presented.  Imaging of barium atoms in solid xenon is demonstrated with sensitivity down to the single atom level.  Achievement of this level of sensitivity is a major step toward barium tagging by this method.

%in the search for this decay
\end{abstract}
% \begin{acknowledgements}
Thanks of course goes to my advisor Bill Fairbank 
\end{acknowledgements}
% \maketitle
% \mainmatter
% \input{ch1intro}
% \input{anotherchapter}
% <etc>
% \backmatter
% \bibliographystyle{ieeetr}
% \bibliography{thebib}
% \appendix
% \input{thatappendix}
% \end{document}

\end{document}