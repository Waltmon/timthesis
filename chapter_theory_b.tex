\chapter{Theory}
\label{chapter:theory}
%\label{chap:theory} this doesn't seem to work

Theory relevant to the spectroscopy of single Ba atoms and Ba\textsuperscript{+} ions in SXe matrices is discussed, beginning with the spectroscopy of Ba and Ba\textsuperscript{+} in vacuum in Sec. \ref{sec:vacuum}, with a numerical solution to a 5-level rate equations model in Sec. \ref{sec:model}.  Matrix isolation spectroscopy of metal species in solid noble gas matrices is discussed in Sec. \ref{sec:matrix}.  Finally, fluorescence efficiency and cross section calculations are presented in Sec. \ref{sec:fluorEff}.

\section{Ba/Ba\textsuperscript{+} Spectroscopy in Vacuum}
\label{sec:vacuum}

Some of the low-lying energy levels for Ba in vacuum are shown in Fig. \ref{fig:elevsBa}.  Transition rates for the lowest-lying Ba energy levels in vacuum are shown in Table \ref{table:BaTransitions}.  The main transition is between the ground $6s^{2}$ $^{1}$S$_{0}$ state to the excited $6s6p$ $^{1}$P$_{1}$\textsuperscript{o} state.  Decays from the $^{1}$P$_{1}$\textsuperscript{o} state to three metastable D states have a branching ratio of about 1 in 330 excitations.  Decays back to ground from the D states are parity-forbidden, resulting in very low decay rates of around 4~s$^{-1}$ for the $^{1}$D$_{2}$ state, and around 0.01~s$^{-1}$ for the $^{3}$D states.  This can lead to significant optical pumping of these states during multiple cycles of excitation of the  $^{1}$S$_{0} \rightarrow ^{1}$P$_{1}$\textsuperscript{o} transition.

%, including that of $^{3}$P$_{1}$\textsuperscript{o} to ground.  Though decays into the $^{3}$P states from $^{1}$P$_{1}$\textsuperscript{o} are parity-forbidden, they may occur for Ba in a SXe matrix, in which case $^{3}$P$_{1}$\textsuperscript{o} to ground is relevant

\begin{figure} %[H]
	\includegraphics[width=.8\textwidth]{figures/elevs_Ba_extra_with3P1.png}
	\caption{Ba energy levels in vacuum.}
    \label{fig:elevsBa}
\end{figure}

The low-lying energy levels of Ba\textsuperscript{+} in vacuum are shown in Fig. \ref{fig:elevsBaPlus}.  Two strong transitions exist between the ground $6s$ $^{2}$S$_{1/2}$ state and the $6p$ $^{2}$P$_{1/2}$ and $6p$ $^{2}$P$_{3/2}$ excited states.  Branching ratios to the two metastable $^{2}$D states result in a decay into a $^{2}$D state after about 4 excitations.  Decay out of the $^{2}$D states takes tens of seconds.  Transition rates between these levels are listed in Table \ref{table:BaTransitions}.    

\begin{table}[!htbp]
\caption{Transition rates for the lowest-lying energy levels of Ba and Ba\textsuperscript{+} in vacuum.} %not sure what [Small Table], between \caption and {}, w/ no spaces, does
\label{table:BaTransitions}
\begin{tabular}{c|c|c c}
& Transition & Rate ($s^{-1}$) &  \\
\hline
Ba & $6s6p$ $^{1}$P$_{1}$\textsuperscript{o} $\rightarrow 6s^{2}$ $^{1}$S$_{0}$ & $1.19 \times 10^{8}$ & \cite{BaAllowedTransitions} \\
& $6s6p$ $^{1}$P$_{1}$\textsuperscript{o} $\rightarrow 6s5d$ $^{1}$D$_{2}$ & $2.50 \times 10^{5}$ & \cite{BaAllowedTransitions} \\
& $6s6p$ $^{1}$P$_{1}$\textsuperscript{o} $\rightarrow 6s5d$ $^{3}$D$_{2}$ & $1.1 \times 10^{5}$ & \cite{BaAllowedTransitions} \\
& $6s6p$ $^{1}$P$_{1}$\textsuperscript{o} $\rightarrow 6s5d$ $^{3}$D$_{1}$ & $3.1 \times 10^{3}$ & \cite{BaAllowedTransitions} \\
& $6s5d$ $^{1}$D$_{2} \rightarrow 6s^{2}$ $^{1}$S$_{0}$ & $4$ & \cite{Ba1D2and3D1} \\
& $6s5d$ $^{3}$D$_{2} \rightarrow 6s^{2}$ $^{1}$S$_{0}$ & $1.45 \time 10^{-2}$ & \cite{Ba3D2} \\
& $6s5d$ $^{3}$D$_{1} \rightarrow 6s^{2}$ $^{1}$S$_{0}$ & $1.7 \time 10^{-2}$ & \cite{Ba1D2and3D1} \\
%& $6s6p$ $^{3}$P$_{1}$\textsuperscript{o} $\rightarrow 6s^{2}$ $^{1}$S$_{0}$ & {\color{red}???} & {\color{red}???}\\
\hline
Ba\textsuperscript{+} & $6p$ $^{2}P_{3/2}$\textsuperscript{o} $\rightarrow 6s $ $^{2}S_{1/2}$ & $1.11 \times 10^{8}$ & \cite{BaAllowedTransitions} \\
& $6p$ $^{2}P_{1/2}$\textsuperscript{o} $\rightarrow 6s$ $^{2}S_{1/2}$ & $9.53 \times 10^{7}$ & \cite{BaAllowedTransitions} \\
& $6p$ $^{2}P_{3/2}$\textsuperscript{o} $\rightarrow 5d$ $^{2}D_{5/2}$ & $4.12 \times 10^{7}$ & \cite{BaAllowedTransitions} \\
& $6p$ $^{2}P_{3/2}$\textsuperscript{o} $\rightarrow 5d$ $^{2}D_{3/2}$ & $6.00 \times 10^{6}$ & \cite{BaAllowedTransitions} \\
& $6p$ $^{2}P_{1/2}$\textsuperscript{o} $\rightarrow 5d$ $^{2}D_{3/2}$ & $3.10 \times 10^{7}$ & \cite{BaAllowedTransitions} \\
& $5d$ $^{2}D_{5/2} \rightarrow 6s$ $^{2}S_{1/2}$ & $3.8 \times 10^{-2}$ & \cite{BaPlusD52} \\
& $5d$ $^{2}D_{3/2} \rightarrow 6s$ $^{2}S_{1/2}$ & $1.3 \times 10^{-2}$ & \cite{BaPlusD32} \\
\end{tabular}
\end{table}

\begin{figure} %[H]
	\includegraphics[width=.6\textwidth]{figures/elevs_Ba+_separate.png}
	\caption{Ba\textsuperscript{+} energy levels in vacuum.}
    \label{fig:elevsBaPlus}
\end{figure}

%These rates are shown in  along with each of the allowed transitions in Fig. \ref{fig:elevs}.

%The lowest-lying energy levels in vacuum for Ba and Ba\textsuperscript{+} are shown in Fig. \ref{fig:elevs}.  For Ba, the main transition is between the ground $6s^{2}$ $^{1}$S$_{0}$ to the excited $6s6p$ $^{1}$P$_{1}$\textsuperscript{o} state, however branching ratios from the P state result in a decay to one of three metastable D states in about 1 in 330 excitations.  For Ba\textsuperscript{+}, two strong transitions exist between the ground $6s$ $^{2}$S$_{1/2}$ and the $6p$ $^{2}$P$_{1/2}$ and $6p$ $^{2}$P$_{3/2}$ excited states.  Transitions to the two metastable D states are higher than for the atom, resulting in a decay into a D state after about 4 excitations.  These D states are metastable because they are parity-forbidden to decay back to the ground state, resulting in much lower decay rates.  These rates are shown in Table \ref{table:BaTransitions} along with each of the allowed transitions in Fig. \ref{fig:elevs}.

%\begin{figure}[H]
%	\includegraphics[width=.9\textwidth]{figures/elevs.png}
%	\caption{Energy level diagrams for (a) Ba and (b) Ba\textsuperscript{+}.}
%    \label{fig:elevs}
%\end{figure}

%These energy levels and their transition rates are well known, and are documented in the NIST Atomic Spectra Database.

Single atom/ion detection by fluorescence requires many excitation/emission cycles in order to detect an observable number of photons from a single atom/ion.  For Ba and Ba\textsuperscript{+}, in addition to the main excitation laser, lasers may be needed to depopulate the metastable D states once the atom/ion becomes optically pumped into one of them.  This recovery is called re-pumping.  For Ba atoms in vacuum, re-pumping can be accomplished via three infrared lasers at wavelengths 1107.6~nm, 1130.0~nm, and 1500.0~nm for the direct transitions back to the $6s6p$ $^{1}$P$_{1}$\textsuperscript{o} state.  An alternative re-pumping scheme is via excitation to higher-level states which have paths back to the ground state or the $6s6p$ $^{1}$P$_{1}$\textsuperscript{o} state.  A few such higher-level re-pump transitions are shown in Fig. \ref{fig:elevsBa}, including two red transitions to the $5d6p$ $^{3}$D$_{1}$\textsuperscript{o}, and blue and violet transitions to the $6s7p$ $^{1}$P$_{1}$\textsuperscript{o} and $6s8p$ $^{1}$P$_{1}$\textsuperscript{o} states.  Trapping and detection of Ba atoms in vacuum in a magneto-optical trap (MOT) was achieved in \cite{BaMOT}.  This was done with two separate re-pumping schemes:  by incorporating three infrared re-pump transitions, as well as with two infrared re-pump transitions along with a 659.7-nm laser re-pumping the $6s5d$ $^{3}$D$_{1}$ state via the $5d6p$ $^{3}$D$_{1}$\textsuperscript{o} state.  Observation of trapped single Ba\textsuperscript{+} ions in vacuum has been achieved using 493.4~nm along with one re-pump laser at 649.7~nm \cite{BaPlus1978,singleBaPlusEXO}.

%, replacing the 1130.6-nm transition directly to $6s6p$ $^{1}$P$_{1}$\textsuperscript{o}

\section{5-level Rate Equations Model}
\label{sec:model}

A 5-level rate equations model for Ba transitions in vacuum, with excitation to the $6s6p$ $^{1}$P$_{1}$\textsuperscript{o} state, is shown schematically in Fig. \ref{fig:5lev}.  States 1 and 2 represent the $6s^{2}$ $^{1}$S$_{0}$ and $6s6p$ $^{1}$P$_{1}$\textsuperscript{o} states, respectively, and states 3, 4 and 5 represent the $6s5d$ $^{1}$D$_{2}$, $6s5d$ $^{3}$D$_{2}$, and $6s5d$ $^{3}$D$_{1}$ states, respectively.  The rate equations for this system, neglecting stimulated emission, are given in Eq. \ref{eqn:rateEqn}:


\begin{figure} %[H]
        \centering
                \includegraphics[width=.4\textwidth]{figures/5-level_general.png}
                \caption{5-level system diagram with excitation rate $W_{12}$ and decay rates $A_{ij}$.}
\label{fig:5lev}
\end{figure}

% is stated in Eqn. \ref{eqn:rateEqn}:

\begin{equation}
\begin{aligned}
\frac{dN_1}{dt} &= - W_{12}N_{1} + A_{21}N_{2} + A_{31}N_{3} + A_{41}N_{4} + A_{51} N_{5} \\
\frac{dN_2}{dt} &= W_{12}N_{1} - N_{2}(A_{21} + A_{23} + A_{24} + A_{25}) \\
\frac{dN_3}{dt} &= A_{23}N_{2} - A_{31}N_{3} \\
\frac{dN_4}{dt} &= A_{24}N_{2} - A_{41}N_{4} \\
\frac{dN_5}{dt} &= A_{25}N_{2} - A_{51}N_{5} \\
\end{aligned}
\label{eqn:rateEqn}
\end{equation}

\noindent
where $N_{i}$ is the population in state $i$, $A_{ij}$ is the spontaneous decay rate from state $i$ to state $j$, and $W_{12}$ is the excitation rate:

\begin{equation}
W_{12} =  \frac{\sigma I}{h \nu}
\label{eqn:w12}
\end{equation}

\noindent
where $\sigma$ is the cross section for excitation, $I$ is the laser intensity, and $h \nu$ is the excitation photon energy.

Under some experimental conditions, the assumption can be made that $W_{12}$ is small compared to the rates out of the $^{1}$P$_{1}$\textsuperscript{o} state.  In this case the population in the excited state $N_{2}$ is small compared to the populations of the other states.  Except for very short times, $dN_2/dt$ is also small compared to $W_{12} N_1$ and $N_2 A_{21}$.  For these conditions, there is an effective pumping rate for the D states:

\begin{equation}
W_{1i} = W_{12}\frac{A_{2i}}{A_{21} + A_{23} + A_{24} + A_{25}}
\label{eqn:smallW12pumpingRates}
\end{equation}

\noindent
where $i = 3,4,5$.  The fraction in Eq. \ref{eqn:smallW12pumpingRates} is the respective branching ratio from the excited state.  With this definition, the rate equations for states 3, 4, and 5 can be re-written as:

%, including the pumping rate back to ground

\begin{equation}
\begin{aligned}
\frac{dN_3}{dt} &= W_{13}N_{1} - A_{31}N_{3} \\
\frac{dN_4}{dt} &= W_{14}N_{1} - A_{41}N_{4} \\
\frac{dN_5}{dt} &= W_{15}N_{1} - A_{51}N_{5} \\
\end{aligned}
\label{eqn:smallW12populationRates}
\end{equation}

\noindent
where $N_{1} = 1 - N_{3} - N_{4} - N_{5}$ by conservation of total population.

For a numerical solution of these equations, initially all population is in the ground state ($N_{3} = N_{4} = N_{5} = 0$).  The model calculates the state populations $N_{i}$ at the end of each iterative time step, $t_{\text{step}}$, as:

\begin{equation}
\begin{aligned}
N_{i} = N_{i, previous} + \frac{dN_i}{dt}t_{\text{step}} \\
\end{aligned}
\label{eqn:smallW12populations}
\end{equation}

\noindent
where $i = 3,4,5$, and $dN_i/dt$ is determined at the beginning of each time step according to Eq. \ref{eqn:smallW12populationRates}.  Time steps of $t_{\text{step}} = 1~\mu$s were used.  In order to model data, which is taken in exposure time ($t_{\text{exp}}$) frames, the model iterates through $t_{\text{exp}}/t_{\text{step}}$ steps in each frame, accumulating fluorescence photons emitted at each step ($\Delta S_{\text{step}}$) according to the fluorescence rate times $t_{\text{step}}$:

\begin{equation}
\begin{aligned}
\Delta S_{\text{step}} = W_{12}  N_{1} \frac{A_{21}}{A_{21} + A_{23} + A_{24} + A_{25}}t_{\text{step}} \\
\end{aligned}
\label{eqn:smallW12counts}
\end{equation}

\noindent
To incorporate camera readout time, where laser exposure continues but fluorescence is not observed, populations and derivatives are iterated without adding to total counts.

\begin{figure} %[H]
        \centering
                \includegraphics[width=.7\textwidth]{figures/thesis_modelExamples.png}
                \caption{Modeled photons emitted in 0.5-s exposure frames for three values of $W_{12}$, using vacuum Ba transition rates, normalized to the first point.  A readout time of 0.1~s is used between frames. }
\label{fig:modelExample}
\end{figure}

Examples of modeled normalized photons emitted vs. time are shown in Fig. \ref{fig:modelExample} for three values of $W_{12}$, using the transition rates of Ba in vacuum from Table \ref{table:BaTransitions}.  Decay of fluorescence occurs as atoms are optically pumped into the metastable D states.  For long times the curves reach a steady state value.  This model is compared to bleaching data of Ba in SXe in Sec. \ref{sec:bleach577and591}.

%\begin{equation}
%\begin{aligned}
%\frac{dN_1}{dt} &= - w_{12}N_{1} + a_{21}N_{2} + a_{31}N_{3} %+ a_{41}N_{4} + a_{51} N_{5} + a_{61}N_{6} \\
%\frac{dN_2}{dt} &= w_{12}N_{1} - N_{2}(a_{21} + a_{23} + %a_{24} + a_{25} + a_{26}) \\
%\frac{dN_3}{dt} &= a_{23}N_{2} - a_{31}N_{3} \\
%\frac{dN_4}{dt} &= a_{24}N_{2} - a_{41}N_{4} \\
%\frac{dN_5}{dt} &= a_{25}N_{2} - a_{51}N_{5} \\
%\frac{dN_6}{dt} &= a_{26}N_{2} - a_{61}N_{6}
%\end{aligned}
%\label{eqn:rateEqn}
%\end{equation}

\section{Matrix Isolation Spectroscopy}
\label{sec:matrix}

%\emph{\color{gray}May be good matrix isolation theory references in Ba Spec... yeah, 1 and 2 i think ... shon's 55, 57, 58 are probably good too and maybe overlapping}

In the matrix isolation technique, a species of interest is trapped in a crystal, or ``matrix," of inert atoms/molecules such that it can be studied at leisure.  To be effective, the temperature must be low enough to prevent diffusion of guest atoms, and the guest concentration must be low enough to prevent guest-guest interaction.  The technique was originally proposed and demonstrated in \cite{matrixIso}, where otherwise short-lived species in were studied by isolating them in various solid matrices.  The proposed method of Ba tagging in SXe is a unique application of matrix isolation spectroscopy.  In order to understand the spectroscopic effects which may be encountered due to interaction between the guest Ba atoms and Ba\textsuperscript{+} ions and the host Xe atoms, some effects encountered in systems of metal atoms in rare-gas matrices are discussed.

%Low concentrations of the guest species, ideally at least 1:$10^{3}$ host:guest ratio, prevent guest-guest interactions.  

The leading interaction between a guest atom and the host noble gas atoms is an induced dipole-dipole Van der Waals force.  For a metal atom of group I or II, this interaction is quite different when the atom is in the excited P state vs. the ground S state, resulting in general in different shaped potential energy curves \cite{crepin}.  To illustrate the concept, guesses at the ground and excited state potentials for a BaXe molecule are shown in Fig. \ref{fig:FranckCondon}.  In a cold matrix, the system will be mainly in the ground lattice vibrational state ($\nu =0$) before excitation.  According to the Franck-Condon principle, the strongest electronic excitation will occur to vibrational states whose wavefunctions overlap that of the ground state.  Because this occurs for a band of vibrations, the absorption is broadened.  The peak absorption energy is determined by the difference in potential energy between the ground state and the median vibrational state excited, which in general is not the same as the absorption energy of atoms in vacuum.  In the excited state, rapid decay occurs to the lowest vibrational state, and then a similar broadening in the spontaneous emission energy occurs.  A redshift in the emission is observed relative to the excitation.  Additionally, splitting of orbital degeneracy in the P state can produce triplet structures in the absorption according to the Jahn-Teller effect \cite{jahnteller}.  These features depend on the specific configuration of atoms surrounding the guest, and thus different spectra can be observed from guest atoms occupying different matrix ``sites."  Annealing a matrix after deposition can reveal the relative stability of various matrix sites \cite{crepin}.

%The ground S state of a group I or II atom is spherically symmetric, and being slightly larger in Van der Waals radius than Xe, it is likely to take the place of one or more Xe atoms in the fcc crystal (substitution), vs. existing in between Xe atoms in the crystal structure (interstitial).

%dynamic

%the strength of the Ba-Xe interaction, an induced dipole-dipole Van der Waals force, is quite different when the Ba is in the non-spherically-symmetric excited P state \cite{crepin}.  

% before electronic decay can occur

\begin{figure} %[H]
        \centering
                \includegraphics[width=.7\textwidth]{figures/FranckCondon.png}
                \caption{Illustration of the Franck-Condon Principle resulting in red-shifted emission as well as broadening in absorption (A) and emission (B) due to vibrational modes $\nu$ ($\nu$') in the ground (excited) state.}
\label{fig:FranckCondon}
\end{figure}

As an example, the spectroscopy on Na atoms in solid rare-gas matrices was studied in \cite{matrixNa}.  Spectral features are exemplified in Fig. \ref{fig:matrixNa} for Na atoms in SAr, SKr, and SXe.  Excitation and emission are broadened to 10's of nm.  Two or three different matrix sites are observed in the three matrices, with the largest red-shift in emission observed in SXe.  The triplet structures in excitation spectra are attributed to the Jahn-Teller effect.  The specific matrix configuration associated with each excitation triplet is identified in this paper through theoretical modeling of the Na-matrix interactions.

\begin{figure} %[H]
        \centering
                \includegraphics[width=.7\textwidth]{figures/Na_in_matrices.png}
                \caption{Excitation and emission spectra of Na atoms in SAr, SKr, and SXe.  \cite{matrixNa}}
\label{fig:matrixNa}
\end{figure}

%As as example \emph{\color{gray}is there a more modern example paper you can use?}, the spectroscopy of Ba in solid Ar (SAr) and solid Kr (SKr) matrices was studied in \cite{SAr}.  The example of Ba in SAr is shown in Fig. \ref{fig:BaSAr}.  The absorption is 10s of nm broad with a multi-peak structure, and the emission is broadened to nms and red-shifted.  These were attributed to the main $6s^{2}$ $^{1}$S$_{0} \rightarrow 6s6p$ $^{1}$P$_{1}$\textsuperscript{o} transition and its spontaneous emission. \emph{\color{gray}talk about other aspects.....brian says the triplet is caused by P state degeneracy being split, but I don't see how this makes sense, and also I have the impression that there are multiple explanations like Jahn Teller -- you should have a feel for this....}

%\begin{figure} %[H]
%        \centering
%                \includegraphics[width=.7\textwidth]{figures/Ba_in_SAr.png}
%                \caption{Absorption and emission spectra of neutral Ba in SAr at 10~K.  \cite{SAr}}
%\label{fig:BaSAr}
%\end{figure}

%Energy level transition probabilities can also be affected in a matrix.  If potential energy curves cross each other, non-radiative transitions can become allowed for otherwise forbidden transitions \cite{crepin}.  Spin-forbidden transitions, both radiative and non-radiative, between triplet and singlet states can become more significant for heavy host atoms like Xe \cite{heavyAtom}.  Effects like these could aid in observation of the Ba/Ba\textsuperscript{+}, e.g. by improving decay rates out of metastable D states, or they could reduce detectability, e.g. if a non-radiative decay competes with the fluorescence channel.

%, e.g., spin-forbidden transitions can become more significant via the heavy atom effect [ref...i know you have crepin but maybe you can use crepin's ref, since you need to understand this anyway]

%The detectability of a single atom/ion can depend greatly on [these]...

\section{Fluorescence Efficiency}
\label{sec:fluorEff}

The fluorescence efficiency ($\epsilon_{f}$) is the ratio of total fluorescence photons emitted to photons absorbed.  This can be calculated by Eq. \ref{eqn:flueEff}:

\begin{equation}
\epsilon_{f} = \frac{f}{W_{12} \epsilon_{\text{tot}}}
\label{eqn:flueEff}
\end{equation}
% = \frac{f h \nu}{\sigma I \epsilon_{c}}

\noindent
where $f$ is the number of fluorescence photons observed per atom per second, $W_{12}$ is the excitation rate, and $\epsilon_{\text{tot}}$ is the photon detection efficiency of the system.  $W_{12}$ can be calculated by Eq. \ref{eqn:w12}.  The cross section can be calculated using the shape of an excitation spectrum according to Eq. \ref{eqn:sigma}:

\begin{equation}
\sigma(\nu) = A_{21} \frac{g_2}{g_1} \frac{c^2}{8 \pi \nu^2} g(\nu)
\label{eqn:sigma}
\end{equation}

\noindent
where $A_{21}$ is the atomic transition rate, $g_1$ and $g_2$ are the ground and excited state degeneracies, respectively, and $\nu$ is the photon frequency.  $g(\nu)$ is the normalized line shape, satisfying:

\begin{equation}
\int_{0}^{\infty} g(\nu) d\nu = 1
\label{eqn:g}
\end{equation}

\noindent
Integrating Eq. \ref{eqn:sigma} then yields:

\begin{equation}
\int_{0}^{\infty} \sigma(\nu) \nu^2 d\nu = A_{21} \frac{g_2}{g_1} \frac{c^2}{8 \pi}
\label{eqn:sigma2}
\end{equation}

\noindent
Since the cross section $\sigma(\nu)$ is proportional to the excitation spectrum $E(\nu)$, we can finally write the cross section as a function of $\nu$ according to Eq. \ref{eqn:sigmafinal}:

\begin{equation}
\sigma(\nu) = \frac{A_{21} E(\nu)}{\int_{0}^{\infty} E(\nu') \nu'^2 d\nu'} \frac{g_2}{g_1} \frac{c^2}{8 \pi}.
\label{eqn:sigmafinal}
\end{equation}