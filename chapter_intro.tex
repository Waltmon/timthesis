\chapter{Introduction}

Neutrinos have provided illumination as well as great challenge to physics since their discovery.  They first entered our consciousness through W. Pauli, who proposed in 1930 the existence of a neutral, unobserved particle to explain the apparent violation of energy conservation in beta decay. He admitted that neutrinos (then deemed ``neutrons'' -- what we now know as neutrons had not been discovered yet either) should be difficult to observe experimentally, but also that it seemed unlikely that they would never have been noticed before [ref: Pauli letter, or a following paper?].  As it turns out, they are much more difficult to observe than he predicted; they aren't likely to be noticed without extreme experimental techniques.

A theory formulated in 1933 by E. Fermi for beta decay, including the neutrino, would be the beginnings of weak theory, and the development of the very successful Standard Model of particle physics.  But neutrinos continued to challenge theory with the discovery of non-zero neutrino mass, and they remain at the forefront of our exploration of the universe.

The possibility that neutrinos are Majorana particles makes the search for neutrinoless double beta decay very important for the further development of particle theory.  Majorana formulation can describe the origin of neutrino mass, and possibly explain why the mass is very small via the Seesaw Mechanism [ref].  Observation of neutrinoless double beta decay would simultaneously demonstrate that neutrinos are Majorana particles, as well as give a measurement of the absolute mass itself [ref? this is said later too].

To motivate barium tagging, this chapter outlines the current theory for neutrinos, and then describes the neutrinoless double beta decay experiments EXO-200 and nEXO.

\section{Neutrinos}

Neutrinos are chargeless leptons which only interact via the weak force (and gravity).  There are three known ``flavors'' of neutrinos, each corresponding to one of the three known leptons:  $\nu_{e}$, $\nu_{\mu}$, and $\nu_{\tau}$.  These are the eigenstates in the basis of the weak force, so they are the states in which a neutrino will interact via the weak force.

\subsection{Neutrino Oscillation and Mass}

{\color{red}Neutrinos exhibit mixing between their energy eigenstates and their weak force eigenstates, and these are not the same basis. }  {\color{red}\textbf{not really:}  This means that a flavor eigenstate is not a stationary state} {\color{gray}[the fact is that quark mixing happens, and one theory was that neutrinos would do something similar, and then oscillation is discovered...etc...um...well, they could be a different basis, but apparently they would not oscillate if their masses were all zero. You might want to work through the equations assuming zero mass and see what time-evolving a flavor state does ... could they still have 3 different energy states (i.e., momenta)?]--- }a neutrino which begins as a pure flavor state (as all neutrinos will, coming out of a quantum process involving one of the three leptons) will oscillate into the other two flavors as it evolves in time, i.e. the probability of measuring it to be one of the other two flavors is no longer zero.

{\color{gray}\emph{History:  }The first indications of neutrino oscillation came around 1970 with the Ray Davis Experiment [ref.], which measured the flux (at Earth) of solar electron neutrinos.  The flux measured was quite a bit lower than predicted by solar models, and this became known as the Solar Neutrino Problem.  The discovery of neutrino oscillation in the late 1990s [ref.] solved this problem, as only a fraction of the sun's neutrinos, produced as pure electron neutrinos, would interact as such.}

The very small mass of a neutrino, specifically relative to its momentum, lets one write its Hamiltonian in terms of mass squared differences $\Delta m_{ij}^{2} = m_{i}^{2} - m_{j}^{2}$, where $i$,$j$ = 1,2,3, referring to what we then call mass states.  The mass basis is really the energy basis with the small mass approximation, along with dropping some constant terms in the Hamiltonian (which do not affect time evolution).  Writing the time evolution in terms of mass squared differences means that neutrino oscillation experiments can produce measurements of these differences.  In fact, the discovery of neutrino oscillation was the first (and only, so far) demonstration that neutrinos have a non-zero mass.  Without neutrino mass (particularly without differences between the masses of the mass states), neutrinos would not oscillate.

Neutrino oscillation experiments also provide measurements on the amount of mixing between the flavor basis and the mass basis.  We define the mixing between them by a rotation in terms of three mixing angles, $\theta_{12}$, $\theta_{23}$, and $\theta_{13}$.  Transformation between the flavor and mass bases is done with the following unitary matrix, called the Pontecorvo--Maki-–Nakagawa–-Sakata (PMNS) matrix:

\begin{equation}
\begin{aligned}
U &= \begin{pmatrix}
1 & 0 & 0 \\
0 & c_{23} & s_{23} \\
0 & -s_{23} & c_{23} \end{pmatrix}
\begin{pmatrix}
c_{13} & 0 & s_{13} e^{-i \delta} \\
0 & 1 & 0 \\
-s_{13} e^{i \delta} & 0 & c_{13} \end{pmatrix}
\begin{pmatrix}
c_{12} & s_{12} & 0 \\
-s_{12} & c_{12} & 0 \\
0 & 0 & 1 \end{pmatrix}
\begin{pmatrix}
1 & 0 & 0 \\
0 & e^{i \alpha_{1}/2} & 0 \\
0 & 0 & e^{i \alpha_{2}/2} \end{pmatrix} \\
& = \begin{pmatrix}
c_{12} c_{13} & s_{12} c_{13} & s_{13} e^{-i \delta} \\
-s_{12} c_{23} - c_{12} s_{23} s_{13} e^{i \delta} & c_{12} c_{23} - s_{12} s_{23} s_{13} e^{i \delta} & s_{23} c_{13} \\
s_{12} s_{23} - c_{12} c_{23} s_{13} e^{i \delta} & -c_{12} s_{23} - s_{12} c_{23} s_{13} e^{i \delta} & c_{23} c_{13} \end{pmatrix}
\begin{pmatrix}
1 & 0 & 0 \\
0 & e^{i \alpha_{1}/2} & 0 \\
0 & 0 & e^{i \alpha_{2}/2} \end{pmatrix}
\end{aligned}
\label{eqn:umatrix}
\end{equation}

\noindent
where $c_{ij} = \cos \theta_{ij}$ and $s_{ij} = \sin \theta_{ij}$.  $\delta$ is a phase factor related to lepton CP violation, and $\alpha_{i}$ are Majorana phases.

Studying oscillations of neutrinos from different kinds of sources, with different energies and path lengths, can isolate sensitivities to the different parameters {\color{gray}(not really sure if this is the right thing to say)}.  For example, the study of solar neutrinos (neutrinos emanating from nuclear fusion reactions in the core of the sun) provides sensitivity to $\theta_{12}$ and $\Delta m_{12}^{2}$ {\color{gray}(\emph{right? $\theta_{12}$ may not be specifically solar...})}.  Beamline neutrino detectors can be designed for maximum sensitivity to parameters.  The parameters so far measured are as follows in Table  \ref{table:nu_osc_vals}:

\begin{table}[!htbp]
\caption{up to date values with references, and denote ``solar'', ``atmos.'', etc.} %not sure what [Small Table], between \caption and {}, w/ no spaces, does
\label{table:nu_osc_vals}
\begin{tabular}{c|c}
Parameter & Measurement \\
\hline
$\Delta m_{12}^{2}$ & \\
$|\Delta m_{31}^{2}|$ & \\
$\sin^{2} \theta_{12}$ & \\
$\sin^{2} \theta_{23}$ & \\
$\sin^{2} \theta_{13}$ & \\
\end{tabular}
\end{table}

Note that only the absolute value of $\Delta m_{31}^{2}$ is known.  As a consequence, there are two possibilities for the hierarchy of the three neutrino masses.  These are called the Normal and Inverted Hierarchies, as shown in Fig. \ref{fig:numasshier}.

\begin{figure}[H]
        \centering
                \includegraphics[width=.5\textwidth]{figures/hierarchy_alterred.png}
                \caption{The two possible hierarchies of neutrino masses.  The colors depict the mixing between the mass and flavor bases. {\color{red}\textbf{[ref]}}}
\label{fig:numasshier}
\end{figure}

\noindent
The correct mass hierarchy remains unknown, but next-generation neutrino experiments, possibly including nEXO, will be able to discern this.

Neutrino oscillation demonstrates that neutrinos have non-zero mass, and though oscillation experiments can measure the mass squared differences, we still do not have a measurement of the absolute masses of the three neutrinos.  {\color{gray}\emph{History:  }mention Pauli's first statement of limit near electron mass?  Then maybe say the 0 assumption, which I think came from Fermi's beta decay theory, but if you mention that above, just refer to it here.}

The current upper limits on the mass come from...(KATRIN(\rightarrow\nu_{e})?  Cosmology(\rightarrow sum)?

Neutrinoless double beta decay experiments like EXO-200 can put upper limits on specifically the Majorana neutrino mass (i.e., upper limits on the neutrino mass if neutrinos are indeed Majorana particles).  As discussed in the next chapter, [EXO-200 and KamLAND (sp?) ZEN (sp?) together provide the strongest Majorana neutrino mass upper limit of [] (IS THAT TRUE?)].

Neutrino oscillation and non-zero neutrino mass are physics beyond the Standard Model (SM) of particle physics, and though much has been discovered through oscillation experiments, there is much yet to learn about neutrinos. Since they are chargeless, they may be Majorana particles, and their small mass could be explained by the See-saw Mechanism [ref]. Majorana particles are their own anti-particle, and this, along with the discovery that neutrinos have mass, allows for a unique test of the Majorana (vs. Dirac) nature of neutrinos: neutrinoless double beta decay.

\subsection{Neutrinoless Double Beta Decay}

Double beta decay is the simultaneous emission of two electrons from a nucleus.  Two-neutrino double beta decay, shown in Fig. \ref{fig:feynman_diags}(left), is allowed by the Standard Model and has been observed in several isotopes which are listed in Table (table).  Similar to beta decay, a neutrino accompanies each electron in this decay, broadening the spectrum of the summed electron energy. This is a second-order process, making it a rare decay, and requiring low backgrounds to measure.

\begin{figure}[H]
        %\centering
        \begin{subfigure}
                \includegraphics[width=.35\textwidth]{figures/feynman_2nu_quarks.png}
                %\caption{barf}
%        %\end{subfigure}
        %\begin{subfigure}
                \includegraphics[width=.35\textwidth]{figures/feynman_0nu_quarks.png}
                \caption{Two-neutrino (left) and Neutrinoless (right) double beta decay.}
        \end{subfigure}
        \label{fig:feynman_diags}
\end{figure}

Neutrinoless double beta decay, shown in Fig. \ref{fig:feynman_diags}(right), is a postulated mode of double beta decay. In this case, the neutrino is exchanged as a virtual particle (which would require that it is a Majorana particle), and there are no neutrinos in the final products. If discovered, not only would neutrinos be determined Majorana particles, but their absolute mass could also be measured in the form of an effective electron neutrino mass, since the rate of neutrinoless double beta decay will depend on the absolute neutrino mass as shown in Eqn. \ref{eqn:rate_vs_mass}:

\begin{equation}
T_{1/2}^{0\nu} = (G^{0\nu}(Q,Z)|M^{0\nu}|^{2}\braket{m_{\nu}}^{2})^{-1}
\label{eqn:rate_vs_mass}
\end{equation}

\noindent
where $T_{1/2}^{0\nu}$ is the $0\nu\beta\beta$ half-life,  $G^{0\nu}$ is a known phase space factor, and $M^{0\nu}$ is a model-dependent nuclear matrix element. The effective electron neutrino mass $\braket{m_{\nu}}$ is the expectation value of the mass for a pure electron neutrino:

\begin{equation}
\braket{m_{\nu}} = \sum\limits_{i} U_{ei}^{2} m_{i}.
\label{eqn:effectivemass}
\end{equation}

The sum of the energies of the emitted electrons in double beta decay will serve as the distinction between the two-neutrino and zero-neutrino modes, shown in Fig. \ref{fig:spectrum_bb}. In the two-neutrino mode, the total decay energy is shared probabilistically between the electrons and the neutrinos (the nucleus recoil energy is negligible), resulting in a broad distribution in the summed electron energy. (Recall the similarly broad electron energy in single beta decay, which ultimately led to discovery of the neutrino involved.) But in the zero-neutrino mode, all of the decay energy is carried away by the two electrons, resulting in only a single allowed value for the summed electron energy -- a peak in the summed electron energy spectrum at the Q-value. 

\begin{figure}[H]
        \centering
                \includegraphics[width=.7\textwidth]{figures/spectrum_bb.png}
                \caption{Conceptual two-neutrino (blue) and zero-neutrino (red) double beta decay spectra.}
\label{fig:spectrum_bb}
\end{figure}

The rarity of double beta decay (see the very long half lives in Table (table)) requires very low backgrounds, especially around the Q-value for the $0\nu\beta\beta$ search. The next sections describe EXO-200 and it's next-generation successor, nEXO.

\section{Enriched Xenon Observatory}

The Enriched Xenon Observatory (EXO) is a set of two experiments, each a LXe time projection chamber (TPC) designed to study the double beta decay of the isotope \textsuperscript{136}Xe, and ultimately to search for the zero-neutrino mode.  There are several advantages to a LXe detector.  Xe is extremely transparent, and scintillates at [around?] [xxx] nm, which is [efficiently collected by [type that the APDs are]] [reference]; so the Xe acts as a detection medium in addition to being the source of the double beta decay [reference? I didn't make up that kind of sentence].  Xe can be continuously purified to maintain large electron lifetimes in the LXe.  Also, the ratio between observed scintillation light and remaining ionized electrons (drifted from the decay site by the TPC's electric field) exhibits a well-known microscopic anti-correlation [ref.], the understanding of which improves the energy resolution of the detector.  Finally, a LXe TPC approach offers the opportunity, [fairly] unique in double beta decay, to reach in and identify, or ``tag'', the daughter Ba\textsuperscript{++} at the site of the double beta decay event, which would provide a background-free identification of neutrinoless double beta decay. Barium tagging is the focus of our group at CSU and is the subject of this thesis.

EXO-200 is the first of the two experiments, and has been operational since April of 20[xx](?).  It is a liquid xenon TPC designed to probe Majorana neutrino masses down to around 100~meV.  [EXO instrum. paper part I]  The following sections decribe the EXO-200 experiment, as well as nEXO, the next-generation tonne-scale liquid xenon TPC which is now in the design stages.  EXO-200 does not have barium tagging implemented, but it is hoped that nEXO will. 

\subsection{EXO-200}



\subsection{nEXO}

\noindent
{\color{gray}\textbf{old stupid intro:}

The study of the neutrino has required extremes in experimental technique from the beginning.  Neutrinos were described by W. Pauli, who first proposed their existence to remedy an apparent violation of energy conservation in beta decay, as being [impossible to detect] [ref.].  Rather, it requires a great deal of sensitiviy, ingenuity, and hardship (just "ingenuity and hardship"?  sensitivity may be redundant) to observe them, and it was [] years before they were first observed by [Reines and Cowan] in [], by [] [ref.].  (is "rather, ..." too demoting-sounding?  it is absolutely not meant to be, of course)

Neutrino experiments of greater discovery power have been developed around the world, and command large collaborations of scientists.  ummmmmmmm  this is supposed to kind of allude to barium tagging as an extreme technique

Neutrinoless Double Beta Decay experiments like EXO are a different kind of neutrino experiment, not detecting neutrinos directly, but searching for an effect (neutrinoless double beta decay itself) which would demonstrate the Majorana nature of neutrinos.  A liquid xenon experiment like EXO provides a the challenging opportuniy for another extreme experimental technique, barium tagging, where a single barium ion would be observed at a specific double beta decay site in the volume.  This thesis is part of an exploration of one promising barium tagging technique.  (these things may be saved for the EXO chapter... idk).

From the first formulation of beta decay theory by E. Fermi [ref.], neutrinos have provided an avenue into a world of new physics, and they continue to be such an avenue.  Questions which may be answered by this up and coming generation of neutrino experiments are expected to help explain how the universe came to be this way.

[lead into barium tagging discussion]

\section{something like "Can We Do This?", but probably not that at all.}

A liquid xenon double beta decay detector allows unique access to the daughters of decays in the liquid volume.  The feasibility of grabbing and detecting a single ion from the volume is what must be determined next.}