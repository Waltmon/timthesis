\noindent
{\color{gray}\textbf{old stupid intro:}

The study of the neutrino has required extremes in experimental technique from the beginning.  Neutrinos were described by W. Pauli, who first proposed their existence to remedy an apparent violation of energy conservation in beta decay, as being [impossible to detect] [ref.].  Rather, it requires a great deal of sensitiviy, ingenuity, and hardship (just "ingenuity and hardship"?  sensitivity may be redundant) to observe them, and it was [] years before they were first observed by [Reines and Cowan] in [], by [] [ref.].  (is "rather, ..." too demoting-sounding?  it is absolutely not meant to be, of course)

Neutrino experiments of greater discovery power have been developed around the world, and command large collaborations of scientists.  ummmmmmmm  this is supposed to kind of allude to barium tagging as an extreme technique

Neutrinoless Double Beta Decay experiments like EXO are a different kind of neutrino experiment, not detecting neutrinos directly, but searching for an effect (neutrinoless double beta decay itself) which would demonstrate the Majorana nature of neutrinos.  A liquid xenon experiment like EXO provides a the challenging opportuniy for another extreme experimental technique, barium tagging, where a single barium ion would be observed at a specific double beta decay site in the volume.  This thesis is part of an exploration of one promising barium tagging technique.  (these things may be saved for the EXO chapter... idk).

From the first formulation of beta decay theory by E. Fermi [ref.], neutrinos have provided an avenue into a world of new physics, and they continue to be such an avenue.  Questions which may be answered by this up and coming generation of neutrino experiments are expected to help explain how the universe came to be this way.

[lead into barium tagging discussion]

section: something like "Can We Do This?", but probably not that at all.

A liquid xenon double beta decay detector allows unique access to the daughters of decays in the liquid volume.  The feasibility of grabbing and detecting a single ion from the volume is what must be determined next.}