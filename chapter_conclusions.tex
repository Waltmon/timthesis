\chapter{Conclusions}

The achievement of single-atom-level imaging of Ba atoms in SXe is significant progress toward the ability to tag barium daughters in the nEXO LXe TPC.  Improved studies of the spectroscopy of Ba in SXe have been presented and utilized in reaching this level with the 619-nm fluorescence, as well as in imaging the 577- and 591-nm fluorescence for deposits down to $\leq$ around $10^{3}$ atoms.

%neutralized Ba\textsuperscript{+} deposits in SXe have been presented.  Of particular interest was temperature and annealing effects, as well as bleaching effects of the 577-, 591-, and 619-nm fluorescence peaks.  Fluorescence was maximal in deposits made around 50~K and observed at 11~K.  The 577- and 591-nm peaks experience similarly rapid bleaching.  A model applied to bleaching data of the 591-nm peak demonstrated agreement with optical pumping rates of Ba in vacuum near the beginning of the bleaching process, with subsequent deviation.  The 619-nm peak experiences much lower bleaching rates.

%Understanding of these effects aided in the design of imaging experiments.  Imaging of Ba atoms in SXe in a focused laser region was demonstrated down to the $\leq 10^{3}$-atom level using the 577- and 591-nm peaks, and down to the single-atom level using the low-bleaching 619-nm peak.

\section{Future Steps}

The next major demonstration will be imaging separate Ba atoms in SXe in scanned images.  As indicated in Sec. \ref{sec:scanning}, overcoming the challenge of variable surface background emission is the first priority.  Studies of fluorescence lifetimes of the 619-nm signal vs. the background using a pulsed laser and a photo-multiplier tube for fast detection will be explored, in order to determine whether a photon-sorting technique may remove background.  The atom counting inherent in scanned images will provide the first measurement of the percentage of deposited Ba\textsuperscript{+} ions visible via the 619-nm fluorescence.

Exploration of re-pumping wavelengths for the 577- and 591-nm peaks, as well as the 570- and 601-nm peaks, will be done using tunable lasers in the infrared.  The ability to count atoms emitting with these different wavelengths would be very valuable.

ions

Talk about Adam's work, show pics and stuff.