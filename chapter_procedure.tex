\chapter{Method}

\section{Wavelength Calibration}

Wavelength calibration of the spectrometer was done using three lasers whose wavelengths were first measured with a Burleigh Wavemeter:  a red diode laser at 656.99~nm, a doubled Nd:YAG laser at 532.23~nm, and the C480 blue dye laser typically around 475~nm.  These lasers were directed at the same position on the sapphire window, and their scatter was imaged along the same path as the Ba fluorescence.  The WinSpec software applies the diffraction grating equation to calibrate each CCD pixel to a wavelength.

\section{Fitting of Spectra}
\label{sec:fitting}

Fitting of fluorescence spectra, with sums of peak-specific fit functions, was used to track peak heights in excitation spectra, annealing cycles, and bleaching.  The center and width parameters of the functions were kept fixed, while the heights were the free fitting parameters.  Center and width parameters for each peak were determined by fitting spectra where that peak is relatively large.  Some fine-tuning was done to match slightly different shapes resulting from different excitation wavelengths.

\begin{figure} %[H]
        \centering
                \includegraphics[width=.5\textwidth]{figures/spectra_fit_a.png}
                ~
                \includegraphics[width=.5\textwidth]{figures/spectra_fit_b.png}
                \includegraphics[width=.5\textwidth]{figures/spectra_fit_c.png}
                ~
                \includegraphics[width=.5\textwidth]{figures/spectra_fit_d.png}
                \includegraphics[width=.5\textwidth]{figures/spectra_fit_e.png}
                ~
                \includegraphics[width=.5\textwidth]{figures/spectra_fit_f.png}
                \caption{Example fits to spectra with excitation at (a) 566.6~nm, (b) 563.4~nm, (c) 546.3~nm, (d) 561.0~nm, (e) 555.9~nm, and (f) 567.3~nm.  Laser scatter can be seen in the lower wavelengths for some figures, especially in (a) where it is on the edge of the Raman filter cutoff.}
\label{fig:specFitsGrn}
\end{figure}

Example fits for several different excitation wavelengths are shown in Fig. \ref{fig:specFitsGrn}.  To incorporate the tail in the shape of the 577- and 591-nm peaks, an asymmetric function of the form $A(1+$erf$(\frac{x-a}{\sigma_{1}})(1-$erf$(\frac{x-a}{\sigma_{2}}))$ was used, where $a$ is the fixed center-defining parameter, $\sigma_{1}$ and $\sigma_{2}$ are fixed left and right width parameters, and $A$ is the free amplitude parameter.  The function erf() is an error function.  The 570-, 601-, 619-, and 670-nm peaks are fit with Gaussian functions with fixed widths ($\sigma$) of 1.7~nm, 4.7~nm, 5.3~nm, and 6.7~nm, respectively.  Rather than attempting frame-by-frame background subtractions, additional Gaussians are fit to the broad and sharp background fluorescence.  Two broad Gaussians centered at around 590~nm and 702~nm, and one sharp Gaussian at 694~nm, are chosen by fitting spectra of Xe-only deposits.  These backgrounds and their excitation spectra are discussed in \ref{sec:bgs}.  The full fit, i.e. the sum of each contributing peak fit, is the dotted black line.  Though the shapes do not match perfectly for all excitation wavelengths, the fits still follow the peak amplitudes well.  Ba emission and excitation spectra results are discussed in \ref{sec:fluorescence}.

%another sprectrum showing partial 601 nm is 20141107 run95.91 (547.0 nm excit.)

%i think you can leave this out: Curves in (a) have about 100$\times$ the laser power used in the excitation spectrum (e.g. (b)) where low intensity is desired to avoid bleaching during the scan.  

%A 566~nm Raman filter was used to attenuate the majority of the laser scatter, however the small amount of scatter passed by the filter was used to determine the frame's excitation wavelength.  Each peak's fit contribution was then integrated and scaled by the frame's laser power, as the power output is not constant through the dye range.

\section{Background Reduction}
\label{sec:bgs}

\begin{figure} %[H]
        \centering
                \includegraphics[width=.5\textwidth]{figures/surfaceBG_a.png}
                ~
                \includegraphics[width=.5\textwidth]{figures/surfaceBG_b.png}
                \caption{(a) Surface background emission spectrum w/ excitation at 570.5~nm, and (b) excitation spectrum through wavelengths in R6G dye range.  The sharp drop in (a) around 608~nm is the Raman filter cutoff.}
\label{fig:surfBG}
\end{figure}

One source of background emission was observed from the surfaces of the window.  Its broad fluorescence is shown in Fig. \ref{fig:surfBG}(a) with a 610-nm Raman filter cutoff and 570.4~nm excitation, and its excitation spectrum is shown in Fig. \ref{fig:surfBG}(b) over the R6G dye range.  The nature of this emission has not been determined, however a few features were identified.  One was that the emission increased as the window temperature was decreased, down to about 100~K where it remained flat down to 11~K, shown in Fig. \ref{fig:BGtempDependence}.  Another feature of the surface background is that it bleaches with laser exposure.  In order to reduce this background in imaging experiments, as well as to reduce run-to-run variation in the background due to bleaching, the sapphire window was pre-bleached for at least half an hour.  Decay of the surface background emission, with intermittent observation during a pre-bleaching process, is shown in Fig. \ref{fig:surfBGbleach}.  For efficient pre-bleaching, the dye laser tuned to 580.5~nm for higher laser power, though the observation points in Fig. \ref{fig:surfBGbleach} are taken with the dye laser at 570~nm with the same laser power used in the following Ba imaging experiment.  During imaging experiments, frequent Xe-only deposits were made in order to track surface background emission to establish proper background subtraction.

%\begin{figure} %[H]
%        \centering
%                \includegraphics[width=.4\textwidth]{figures/xe_variation.png}
%                \caption{Background counts in focused laser region over the course of an experiment.}
%\label{fig:xevar}
%\end{figure}

%\emph{\color{gray}from results} Variation in the background level, dominated by the surface background, is shown in Fig. \ref{fig:xevar}.  Variation was most likely caused by drift of the laser position on the window, to regions of different historical bleaching.  Local variations are at the single-atom signal level, however positive signal after subtraction, even at the single-atom level, demonstrates that this variation is sufficiently low.

%Spacial variation in the background is especially cumbersome in a laser scanning experiment, as discussed in \ref{sec:scanning}. 

%speculation:    Given these behaviors, it is possible that the surface background is caused by a species which freezes to the window, or something which coats the window and fluoresces more at lower temperatures.  In either case, the bleaching could be explained by evaporation with laser heating, or by optical pumping of the species into a metastable state.

\begin{figure} %[H]
        \centering
                \includegraphics[width=.6\textwidth]{figures/bg_temp_dep.png}
                \caption{Temperature dependence of surface (red) and sapphire bulk (b) backgrounds.}
\label{fig:BGtempDependence}
\end{figure}

\begin{figure} %[H]
        \centering
                \includegraphics[width=.4\textwidth]{figures/Bleach_SurfaceBG_20150807_part1.png}
                \caption{Decay of surface background emission during pre-bleaching of the sapphire window.}
\label{fig:surfBGbleach}
\end{figure}

\begin{figure} %[H]
        \centering
                \includegraphics[width=.5\textwidth]{figures/Cr_a.png}
                ~
                \includegraphics[width=.5\textwidth]{figures/Cr_b.png}
                \caption{(a) Sapphire bulk emission with 562-nm excitation at 11~K, and (b) excitation spectrum of the sharp 694-nm emission peak using three different laser dyes.  Due to different laser powers and exposure times, the R110 and R6G were scaled to match at their boundary.}
\label{fig:Cr}
\end{figure}

\begin{figure} %[H]
        \centering
                \includegraphics[width=.7\textwidth]{figures/Cr_broad.png}
                \caption{Excitation spectra for weaker sapphire bulk emissions (blue,orange) along with that of the strong 694-nm emission (red).}
        \label{fig:CrBroad}
\end{figure}

Another source of background is fluorescence from the sapphire window bulk.  A spectrum of this fluorescence with 562~nm excitation is shown in Fig. \ref{fig:Cr}(a).  The strong, sharp peak at 694~nm is a well-known \emph{\color{gray}Bill comments} emission of Cr\textsuperscript{3+} impurities in the sapphire bulk.  An excitation spectrum for this peak is shown in Fig. \ref{fig:Cr}(b) over the range all three dyes R6G, R110, and C480, using a sapphire window with relatively high Cr\textsuperscript{3+} content.  Multiple features observed in the excitation spectrum, obtained by integrating the 694-nm peak fit (Sec. \ref{sec:fitting}) vs. excitation wavelength, are consistent with the absorption spectrum of Cr\textsuperscript{3+} in sapphire at 77~K, including three sharp peaks in the blue at 468.4, 474.8, and 476.5~nm, as well as a broad absorption in the green/yellow, peaking around 550~nm, with vibrational peaks on the red tail \cite{SapphireFord,SapphireMcclure}.  In addition to the 694-nm peak, a weaker and much broader emission is observed, along with three weak peaks around the 619-nm Ba peak region, also from the sapphire bulk.  Excitation spectra for these fluorescence components are shown in Fig. \ref{fig:CrBroad} for the R6G dye range.  In this experiment, the laser was de-focused to about w = 200~$\mu$m, and the emission observed was had contribution of the surface background as well as the bulk sapphire emission.  This is negligible for the prominent 694-nm peak, however the rising features of the surface background emission, near 600~nm and 567~nm, can be seen in the excitation spectra of the weaker components (blue, and especially orange curves in Fig. \ref{fig:CrBroad}).  Nonetheless, observation of the same vibrational peaks as in the 694-nm peak demonstrates that the broad emission and weak peaks in the 620 band-pass (Fig. \ref{fig:Cr}(a)) are also due to Cr\textsuperscript{3+} in the sapphire.  Commercially available c-plane quality sapphire windows contain low concentrations of Cr\textsuperscript{3+}.  Sample windows of 0.75" diameter and 0.02" thickness from a few companies were tested, and those from Meller Optics produced the lowest sapphire bulk emission in the 620-nm band-pass region.

The choice of 570~nm for excitation of the 619-nm fluorescence peak was made by an optimization of $S$/$\sqrt{B}$, where $S$ is the 619-nm signal and the background $B$ is the surface background emission.  This was determined by plotting the ratio between binned versions of the 619-nm and surface background excitation spectra, shown in Fig. 

%The Cr\textsuperscript{3+} emission has an inverse relationship with temperature, shown in Fig. \ref{fig:BGtempDependence}.

%with Cr\textsuperscript{3+} concentrations of around {\color{red}10 ppt} observed.

%(peaks around? May need to ask Bill about what peaks exist and at what temps ... a new idea is to just leave it vague, as ``peaks" since you're sure there are at least 2, but vauge is OK)
