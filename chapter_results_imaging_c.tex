\chapter{Results: Imaging}
\label{chapter:imaging}

%\emph{\color{red}do corr. on p-meter sensitive A ($\times$1.57) and 89 percent optical flat T}

Results presented in Chapter \ref{chapter:spectroscopy} demonstrate the best conditions for imaging small numbers of Ba atoms in SXe.  Optimum excitation wavelengths were determined by studying the excitation spectra of the signal as well as the background.  Bleaching studies determined optimum laser intensities to be used.  Deposits made at $50 \pm 5$~K produced more Ba fluorescence signal than those made at 11~K.  An image of Ba atoms emitting at 577 and 591~nm is presented in Sec. \ref{sec:imaging590and577} at the level of $\leq 2.7 \times 10^{3}$ average atoms exposed. Images of Ba\ atoms emitting at 619~nm are presented down to the single-atom level in Sec. \ref{sec:imaging619}.  Initial scanned images of Ba\textsuperscript{+} deposits are presented in Sec. \ref{sec:scanning}.

Though the imaging spectrometer can produce spacial images with the 0-order grating reflection, better collection efficiency and imaging quality are achieved by removing the spectrometer and imaging directly onto the CCD.  Band-pass filters were used to pass the desired Ba fluorescence peak(s) while attenuating laser scatter and sapphire fluorescence.

%angle=90,
\begin{figure} %[H]
        \centering
                \includegraphics[width=.4\textwidth]{figures/imageExamp.png}
                \caption{Example image of focused laser exciting a Ba\textsuperscript{+} deposit on a c-plane sapphire window of 0.5~mm thickness.  A 620-nm band-pass filter passes the fluorescence.}
\label{fig:imageexamp}
\end{figure}

An example image of the focused 570~nm dye laser on a Ba\textsuperscript{+} deposit is shown in Fig. \ref{fig:imageexamp}.  With 4$\times$ magnification, each pixel is 5~$\mu$m$\times$5~$\mu$m.  The laser's path through the window is faintly visible by the broad Cr\textsuperscript{3+} fluorescence in the 620-nm band-pass filter.  The laser is focused at the top surface of the window, which faces the ion beam.  The surface background is seen on the back surface, and the 619-nm Ba fluorescence stands out above both backgrounds on the top surface.  The imaging resolution results in a 1/e$^{2}$ radius of about 12~$\mu$m when imaging the 2.06~$\mu$m $\times$ 2.66~$\mu$m 1/e$^{2}$ laser spot.

%Aberrations and vibrations in the collection optics could contribute to this, as well as cryostat vibrations.

% inability to reach the diffraction limit in imaging.

%\begin{wrapfigure}{r}{0.5\textwidth}
  %\begin{center}
   % \includegraphics[width=0.35\textwidth]{figures/raw_14-atom_labels_from_paper_1f.png}
  %\end{center}
  %\caption{}
  %\label{fig:imageexamp}
%\end{wrapfigure}

\section{Imaging 577- and 591-nm Fluorescence}
\label{sec:imaging590and577}

\begin{figure} %[H]
        \centering
                \includegraphics[width=.6\textwidth]{figures/image_1e4.png}
                \caption{Image of $\leq 2.7 \times 10^{3}$ average Ba atoms in SXe.  100-s exposure with .02~$\mu$W of 566~nm excitation, with laser beam waist of 5~$\mu$m.  Sample was deposited at 50~K and observed at 11~K.  $3 \times 3$ pixel binning was done with software.}
\label{fig:image590s}
\end{figure}

First attempts at imaging small numbers of Ba atoms in a focused laser region were done with the 577- and 591-nm Ba fluorescence peaks together using a 586-nm band-pass filter, which passes 573 - 599~nm FWHM.  This filter has a 2" diameter, resulting in a collection efficiency of $8.4 \times 10^{-3}$, 4$\times$ the efficiency given in Table \ref{table:colleff}.  An image of $\leq 2.7 \times 10^{3}$ atoms on average, with $\leq 8.3 \times 10^{3}$ total atoms exposed, is shown in Fig. \ref{fig:image590s} for a 100-s exposure with 0.02~$\mu$W of 566-nm excitation.  The laser was focused by the bi-convex lens, resulting in a beam waist of 5~$\mu$m as discussed in Sec. \ref{sec:laser}.  With this spot size, the factor between total and average atoms exposed is 3, as discussed in Sec. \ref{sec:vibes}.  At this low intensity, no bleaching was observed in the four frames observed (frame 1 is shown).  Groups of 9 ($3 \times 3$) CCD pixels have been binned in software to produce the nice peak shown.  As discussed in \ref{sec:bleaching}, detection of very small numbers of atoms in these sites may require several re-pump lasers.  As a result of low total exposure, neither the sapphire nor the surface backgrounds are present in these images, though a Xe-only image has been subtracted. 

%$5 \times 10^{3}$        $\leq$ 10$^{4}$

%(10$^{4}$ Ba\textsuperscript{+} ions deposited into the effective laser region)

%Bleaching data was used to optimize laser intensity and exposure time to achieve maximal signal in frame 1, and to avoid hole bleaching.  The fast bleaching of these peaks is the limiting factor on sensitivity.

\begin{figure} %[H]
        \centering
                \includegraphics[width=.95\textwidth]{figures/xebaxe_average_scrunched.png}
                \caption{Raw images of three Ba\textsuperscript{+} deposits yielding (a) $\leq 49$, (b) $\leq 4$, and (c) $\leq 1$ average number of Ba atoms, with their preceding and succeeding Xe-only deposits.  Exposures are 60~s with around 0.24~mW of 570~nm excitation focused to w$_{\text{x}} \times$ w$_{\text{y}} =$ 2.06~$\mu$m $\times$ 2.66~$\mu$m.}
\label{fig:xebaxe}
\end{figure}

\section{Imaging 619-nm Fluorescence}
\label{sec:imaging619}

%This paragraph is kind of dumb and maybe not even necessary:
%At low laser intensities, the 619-nm peak appears minor compared to the 577- and 591-nm peaks.  However, its much lower bleaching rate brings it to dominate over larger exposures, e.g. with the intensity of a focused laser at 0.01-0.1~mW over seconds or minutes.  Experiments imaging small numbers of Ba atoms in SXe in a focused laser region with the 619-nm peak were successful down to the single-atom level.

%\begin{figure} %[H]
%        \centering
%                \includegraphics[width=.4\textwidth]{figures/xe_variation.png}
%                \caption{Background counts in focused laser region over the course of an experiment.}
%\label{fig:xevar}
%\end{figure}

Raw images of Ba\textsuperscript{+} deposits and their preceding and succeeding Xe-only deposits are shown in Fig. \ref{fig:xebaxe} for deposits of (a) $\leq 49$ ($\leq 230$), (b) $\leq 4$ ($\leq 20$), and (c) $\leq 1$ ($\leq 5$) average (total) Ba atoms.  The aspherical laser focusing lens and astigmatism compensator were used for a w$_{\text{x}} \times$ w$_{\text{y}} =$ 2.06~$\mu$m $\times$ 2.66~$\mu$m 1/e$^{2}$ laser region, and thus the factor between total and average atoms exposed is 4.7.  Signal is distinguishable from background by eye, even at the 1-atom average signal level.

%An imaging experiment consisted of many different pulsed Ba\textsuperscript{+} deposits, each of which was evaporated after observation.  This procedure is described in \ref{sec:deposition}-\ref{sec:collection}.

\begin{figure} %[H]
        \centering
                \includegraphics[width=.5\textwidth]{figures/lin_20150526and20150807_log.png}
                ~
                \includegraphics[width=.5\textwidth]{figures/lin_20150526and20150807_lin.png}
                \caption{619-nm signal, scaled by laser intensity and exposure time, vs. average number of Ba\textsuperscript{+} exposed by focused laser at 570~nm for experiments with two different laser focusing lenses, on log (left) and linear (right) scales.} %:  bi-convex (black) and asphere (orange).
\label{fig:lin}
\end{figure}

\begin{figure} %[H]
        \centering
                \includegraphics[width=.99\textwidth]{figures/train.png}
                \caption{Subtracted images of 619-nm fluorescence in focused laser region for runs near linear trend line in signal vs. ions deposited.  Exposures are 60~s with around 0.24~mW of 570~nm excitation focused to w$_{\text{x}} \times$ w$_{\text{y}} =$ 2.06~$\mu$m $\times$ 2.66~$\mu$m.}
\label{fig:train}
\end{figure}

619-nm fluorescence signal is plotted vs. average number of ions (upper limit on atoms) exposed in Fig. \ref{fig:lin}, on log and linear scales, for many deposits in two experiments with two different laser focusing lenses.  The bi-convex laser-focusing lens (black), with no astigmatism compensation, results in a laser waist of $w = 5~\mu$m, while the aspherical lens with astigmatism compensation (orange) results in w$_{\text{x}} \times$ w$_{\text{y}} =$ 2.06~$\mu$m $\times$ 2.66~$\mu$m.  Laser powers were around 2~mW with the bi-convex lens and 0.24~mW with the asphere, resulting in similar intensities.  The signal was determined by summing the counts of a 3-pixel$\times$3-pixel (15$\mu$m$\times$15$\mu$m) region centered on the laser spot in the image, and background was determined by averaging the 3-pixel$\times$3-pixel sum of the Xe-only runs before and after each Ba\textsuperscript{+} run.  Signal and background were scaled to laser intensity and exposure time, and then background was subtracted.  The two experiments demonstrate a linear relationship between signal and number of ions exposed over more than two orders of magnitude.  The linear scale includes a zoom into small numbers of ions in the inset.  Zero ion deposits are produced by retracting the Faraday cup for 1~s without pulsing the ion beam.  The slope of the linear fit, which does not include a constant value, gives about $131 \pm 6$~counts/(mW/$\mu$m\textsuperscript{2} s) per atom.  Deposits at the single-ion level average to $800 \pm 240$~counts/(mW/$\mu$m\textsuperscript{2} s), higher than the linear trend line.

Images of 619-nm Ba fluorescence are shown in Fig. \ref{fig:train} for a few deposits near the linear trend line.  Each has subtracted an image of a nearby Xe-only deposit so that only the Ba fluorescence is seen.  Clear, sharp peaks are observed from an average number of atoms all the way down to the single-atom level, with no clear peak in the zero-atom deposit (where the Faraday cup is opened but the ion beam is not pulsed).  This result is a major step toward the ultimate success of Ba tagging, and is generating excitement and interest in the nEXO collaboration.

\begin{figure} %[H]
        \centering
                \includegraphics[width=.95\textwidth]{figures/xebaxe_larger_average.png}
                \caption{Raw images of a Ba\textsuperscript{+} deposit yielding $\leq 470$ average number of Ba atoms, with its preceding and succeeding Xe-only deposits.  Exposures are 10~s with 0.6~mW of 566~nm excitation focused with bi-convex lens to w = 5~$\mu$m. \emph{\color{red}need to make sure signal is consistent with other stuff or don't use this}}
\label{fig:xebaxeLarger}
\end{figure}

Another important observation is the lack of historical buildup of Ba fluorescence between deposits, even for large Ba\textsuperscript{+} deposits.  This is demonstrated in Fig. \ref{fig:xebaxe}, where the Ba fluorescence is no longer present in the Xe-only runs following each of the deposits.  This is shown further for a deposit 10$\times$ larger in Fig. \ref{fig:xebaxeLarger}.  The stronger presence of laser path through the sapphire window is caused by higher Cr\textsuperscript{3+} content in the window used in this experiment.  The disappearance of Ba signal after evaporation of each sample is important for the implementation of this method of Ba tagging on a probe in nEXO.

%, with a 0-atom intercept of about $820 \pm 460$~counts

%MAYBE: \emph{\color{gray}talk about non-significant non-zero value there?  are the 1-atom also not significant then? ... mention counts / whatev}

%Frequent Xe-only deposits ware made, typically every three runs, to establish the local background.  The image of each deposit was analyzed by summing the counts of a 3-pixel$\times$3-pixel (15$\mu$m$\times$15$\mu$m) region centered on the laser spot in the SXe layer where it excites Ba atoms, i.e. at the top of the laser path in Fig. \ref{fig:imageexamp}.  A partial-bin integrator was implemented such that the 3-pixel$\times$3-pixel region could be centered on the peak center of each run, which typically varied by about half a micron between runs.  Summed counts vs. Ba\textsuperscript{+} ions deposited into the laser region is shown in Fig. \ref{fig:lin} for a combination of data sets from two separate days \emph{\color{blue}maybe you want them separate (with each point having errors) by color to show agreement }  Each point has subtracted the averaged counts from the two surrounding Xe-only deposits.  Some variation in signal is observed, likely due to spatial drifting of the ion beam, as this variation was larger on days where larger beam drift was observed.  Nonetheless, a clear linearity in observed in signal vs. ions deposited, all the way down to single ions deposited.  Recall that the number of ions is an upper limit on the number of atoms in the 619-nm peak matrix site.  ``0-pulse" deposits are also made, wherein Faraday cup 3 is retracted for the 1-s period, but the ion beam remains deflected with no pulses.  This establishes the level of neutral Ba atoms making their way down the beam line from the source into the laser region.  Signal at the few-atom level is observed from these deposits.

%... and try combining those other days.  It will require scaling for intensity difference (spher. abb.) and time, and remember different I$\times$t might give different efficiency.

%\emph{\color{gray}Paragraph on BG variation was here, but I moved it to Method w/o thinking thru completely in this context.  Image is also commented out here, above train figure... and actually I took it out there too...}

%Variation in the background level, dominated by the surface background, is shown in Fig. \ref{fig:xevar}.  Variation was most likely caused by drift of the laser position on the window, to regions of different historical bleaching.  Local variations are at the single-atom signal level, however positive signal after subtraction, even at the single-atom level, demonstrates that this variation is sufficiently low.

%\begin{figure} %[H]
%        \centering
%                \includegraphics[width=.5\textwidth]{figures/lin_20150526.png}
%                ~
%                \includegraphics[width=.5\textwidth]{figures/lin_pres_zoom_predictSingleAtom.png}
%                \includegraphics[width=.5\textwidth]{figures/lin_vsPulses_all_3x3_pre-165.png}
%                ~
%                \includegraphics[width=.5\textwidth]{figures/lin_20150526_cut.png}
%                \caption{\color{gray}this day looks so good that i think you should use it in some way, w/ area corr of course, maybe and ideally combined with lower number stuff -- maybe the variation in e.g. 8-7 is at same level as that in this data? maybe the lin from this will look boss.  remember it has bi-convex.  Showing w/ Xes, as in 3rd thing, is prolly cool too.}
%\label{fig:lintemp}
%\end{figure}

\section{Scanned Images}
\label{sec:scanning}

The ultimate demonstration of single-atom imaging will be to scan the focused laser over samples with more widely	 separated Ba atoms, observing a peak when the laser moves over individual atoms.  The resolution in a scanned image is determined by the laser spot size, even when the imaging optics have a lower resolution.  Preliminary scans were performed by scanning the focusing lens in a raster pattern, using the setup described in Sec. \ref{sec:laserscanning}.  Summed counts from a $3 \times 3$ pixel area around the focused laser, scaled by laser intensity and exposure time, are plotted vs. raster position for several deposits in Fig. \ref{fig:scans}.  These scans are grids of 20 points in X at about 3.75~$\mu$m/step, by 5 points in Y at about 4.78~$\mu$m/step.  An overall higher signal is observed with 48.4 average Ba\textsuperscript{+}/step (c), however single-atom peaks are not distinguished in the sparse deposit of 0.18 average Ba\textsuperscript{+}/step (b).  The aspherical laser focusing lens and astigmatism compensator are used for w$_{\text{x}} \times$ w$_{\text{y}} =$ 2.06~$\mu$m $\times$ 2.66~$\mu$m 1/e$^{2}$ laser region, though cryostat vibrations increase the exposed region as described in Sec. \ref{sec:vibes}.  This had the effect of exposing neighboring step regions to the laser.  Pre-bleaching of the surface background was done with a  of the dye laser at 580.5~nm, in a $22 \times 7$ grid centered on the $20 \times 5$ grid used during the following experiment, with 10~s at each location.

\begin{figure} %[H]
        \centering
                \includegraphics[width=.5\textwidth]{figures/scan_a.png}
                ~
                \includegraphics[width=.5\textwidth]{figures/scan_b.png}
                \includegraphics[width=.5\textwidth]{figures/scan_c.png}
                ~
                \includegraphics[width=.5\textwidth]{figures/scan_d.png}
%                \includegraphics[width=.45\textwidth]{figures/scans_dframeXe.png}
                \caption{Early attempts at scanned images of a few deposits: (a) Xe-only, (b) 0.18 average Ba\textsuperscript{+}/step, (c) 48.4 average Ba\textsuperscript{+}/step, and (d) Xe-only.  10-s exposures with 0.4-0.5~mW of focused 570~nm laser.}
\label{fig:scans}
\end{figure}

\begin{figure} %[H]
        \centering
                \includegraphics[width=.35\textwidth]{figures/scans_dframeXe.png}
                \caption{Distribution of difference in counts ($\Delta_{\text{steps}}$) between successive scan frames in Xe-only deposits.}
\label{fig:scanVarXe}
\end{figure}

\begin{figure} %[H]
        \centering
                \includegraphics[width=.45\textwidth]{figures/lin_withScan_lin.png}
                \caption{619-nm signal, scaled by laser intensity and exposure time, vs. average number of Ba\textsuperscript{+} exposed by focused laser at 570~nm from Fig. \ref{fig:lin} with that of scanned images of Ba\textsuperscript{+} deposits.}
\label{fig:linScan}
\end{figure}

Scans from the data set in Fig. \ref{fig:scans} reveal step-to-step variation in the background, which is dominated by the surface background.  The difference in counts between successive grid steps is shown as a distribution in Fig.\ref{fig:scanVarXe} for all Xe-only deposits in this set of deposits.  Fluctuations occurred at the level of the $800 \pm 240$~counts/(mW/$\mu$m\textsuperscript{2} s) single-atom signal, and above the level of the $131 \pm 6$ counts/(mW/$\mu$m\textsuperscript{2} s) trend line single-atom signal.  Single-atom signals are not expected to stand out under these conditions.  Average counts per step, again scaled by laser intensity and exposure time, are plotted vs. average number of ions per step in Fig. \ref{fig:linScan}, overlain in red crosses on the data from Fig. \ref{fig:lin}, demonstrating signal levels consistent with previous experiments.  The next steps in scanned imaging of Ba atoms are discussed in Sec. \ref{sec:future}.

%\emph{\color{gray}choose what to show and what to say about it -- there are some decent things to show really}  [issues are BG, reproduceability, vibrations]

%\emph{\color{gray}See thought\_process.pptx slide about expected signal / observed BG fluctuation, etc. (unfinished).}