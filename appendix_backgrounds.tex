\chapter{Background Spectra}
\label{appendix:bgs}

The sharp and broad background discussed in Appendix \ref{sec:fitgrn} is mainly due to Cr\textsuperscript{3+} impurity ions in the sapphire window.  A spectrum of this fluorescence with 562~nm excitation is shown in Fig. \ref{fig:Cr}(a).  The strong, sharp peak around 693~nm is a well-known $^{2}E$ - $^{4}A_{2}$ emission in the $d^{3}$ configuration of Cr\textsuperscript{3+} impurities in the sapphire bulk \cite{SapphireRlines1964,SapphireRlines2010}.  An excitation spectrum for this peak is shown in Fig. \ref{fig:Cr}(b) over the range all three dyes R6G, R110, and C480, using one of the sapphire windows with higher Cr\textsuperscript{3+} content.  Multiple features are observed in the excitation spectrum, obtained by integrating the 693-nm peak fit (Appendix \ref{appendix:fitting}) vs. excitation wavelength.  The broad absorption in the green/yellow, peaking around 550~nm, with vibrational peaks on the red tail, agree with well understood features in the spectrum of Cr\textsuperscript{3+} in sapphire, including three sharp peaks in the blue at 468.4, 474.8, and 476.5~nm \cite{SapphireFord,SapphireMcclure}.  In addition to the 693-nm peak, a weaker and much broader emission is observed, along with three weak peaks in the 615-635~nm region.  Excitation spectra for these fluorescence components are shown in Fig. \ref{fig:CrBroad} for the R6G dye range.  In this experiment, the laser was de-focused to about w = 200~$\mu$m, and the emission observed had contribution of the surface background as well as the bulk sapphire emission.  The surface background is negligible for the prominent 693-nm peak.  However, the rising features of the surface background excitation spectrum near 600~nm and 567~nm (see below) can be seen in the excitation spectra of the weaker components (blue, and especially orange curves in Fig. \ref{fig:CrBroad}).  Nonetheless, observation of the same vibrational peaks as in the 693-nm peak excitation spectrum demonstrates that the broad emission and weak peaks in the 620 band-pass (Fig. \ref{fig:Cr}(a)) are also due to Cr\textsuperscript{3+} in the sapphire.  Commercially available c-plane quality sapphire windows contain low concentrations of Cr\textsuperscript{3+}.  Sample windows of 0.75" diameter and 0.02" thickness from a few companies were tested, and those from Meller Optics produced the lowest sapphire bulk emission in the 620-nm band-pass region.

% at 77~K

\begin{figure} %[H]
        \centering
                \includegraphics[width=.5\textwidth]{figures/Cr_a.png}
                ~
                \includegraphics[width=.5\textwidth]{figures/Cr_b.png}
                \caption{(a) Sapphire bulk emission with 562-nm excitation at 11~K, and (b) excitation spectrum of the sharp 693-nm emission peak using three different laser dyes.  Due to different laser powers and exposure times, the R110 and R6G were scaled to match at their boundary.}
\label{fig:Cr}
\end{figure}

\begin{figure} %[H]
        \centering
                \includegraphics[width=.7\textwidth]{figures/Cr_broad.png}
                \caption{Excitation spectra for weaker sapphire bulk emissions (blue,orange) along with that of the strong 693-nm emission (red) at 11~K.}
        \label{fig:CrBroad}
\end{figure}

An additional background emission was observed from the surfaces of the window.  Its broad fluorescence is shown in Fig. \ref{fig:surfBG}(a) with a 610-nm Raman filter cutoff and 570.4~nm excitation, and its excitation spectrum is shown in Fig. \ref{fig:surfBG}(b) over the R6G dye range.  The nature of this emission has not been determined, however a few characteristics were identified.  One was that the emission increased as the window temperature was decreased, down to about 100~K where it remained flat down to 11~K, shown in Fig. \ref{fig:BGtempDependence}.  Another feature of the surface background is that it bleaches with laser exposure.  In order to reduce this background in imaging experiments, as well as to reduce run-to-run variation in the background due to bleaching, the sapphire window was pre-bleached for at least half an hour at 100~K.  Decay of the surface background emission from a focused laser region, with intermittent observation during a pre-bleaching process, is shown in Fig. \ref{fig:surfBGbleach}.  For efficient pre-bleaching, the dye laser was usually tuned to 580.5~nm for higher laser power ($\sim$10~mW), although the bleaching data in Fig. \ref{fig:surfBGbleach} were taken with the dye laser at 570~nm ($\sim$0.1~mW) with the same laser power used in typical Ba imaging experiments.  During imaging experiments, frequent Xe-only deposits were made in order to track the surface background emission to establish proper background subtraction.

\begin{figure} %[H]
        \centering
                \includegraphics[width=.5\textwidth]{figures/surfaceBG_a.png}
                ~
                \includegraphics[width=.5\textwidth]{figures/surfaceBG_b.png}
                \caption{(a) Surface background emission spectrum w/ excitation at 570.5~nm, and (b) excitation spectrum in R6G dye range.  The sharp drop in (a) around 608~nm is the Raman filter cutoff.}
\label{fig:surfBG}
\end{figure}

%\begin{figure} %[H]
%        \centering
%                \includegraphics[width=.4\textwidth]{figures/xe_variation.png}
%                \caption{Background counts in focused laser region over the course of an experiment.}
%\label{fig:xevar}
%\end{figure}

%\emph{\color{gray}from results} Variation in the background level, dominated by the surface background, is shown in Fig. \ref{fig:xevar}.  Variation was most likely caused by drift of the laser position on the window, to regions of different historical bleaching.  Local variations are at the single-atom signal level, however positive signal after subtraction, even at the single-atom level, demonstrates that this variation is sufficiently low.

%Spacial variation in the background is especially cumbersome in a laser scanning experiment, as discussed in \ref{sec:scanning}. 

%speculation:    Given these behaviors, it is possible that the surface background is caused by a species which freezes to the window, or something which coats the window and fluoresces more at lower temperatures.  In either case, the bleaching could be explained by evaporation with laser heating, or by optical pumping of the species into a metastable state.

\begin{figure} %[H]
        \centering
                \includegraphics[width=.6\textwidth]{figures/bg_temp_dep.png}
                \caption{Temperature dependence of surface (red) and sapphire bulk (b) backgrounds. Fluorescence is observed through 620-nm band-pass filter.}
\label{fig:BGtempDependence}
\end{figure}

\begin{figure} %[H]
        \centering
                \includegraphics[width=.4\textwidth]{figures/Bleach_SurfaceBG_20150807_part1_vsTime.png}
                \caption{Decay of surface background emission during pre-bleaching of the sapphire window.}
\label{fig:surfBGbleach}
\end{figure}

\begin{figure} %[H]
        \centering
                \includegraphics[width=.7\textwidth]{figures/S_to_B_both.png}
                \caption{Optimization of signal to background with the 619-nm fluorescence signal (S) for emission from the surface background (B$_{\text{surf}}$, red) and for the bulk sapphire emission (B$_{\text{sap}}$, cyan).}
        \label{fig:StoB}
\end{figure}

Consideration of signal-to-background ($S$/$\sqrt{B}$) guided the choice of 570~nm for excitation of the 619-nm fluorescence.  The signal, background, and $S$/$\sqrt{B}$ for emission passed by the 620-nm band-pass filter is plotted vs. excitation wavelength in Fig. \ref{fig:StoB} for both the surface background (B$_{\text{surf}}$) and the sapphire bulk emission (B$_{\text{sap}}$).  The peak in $S$/$\sqrt{B}$ represents the optimal excitation wavelength respective to each of the two background sources.  Around 568.5~nm is optimal vs. the sapphire emission, and around 571~nm is optimal vs. the surface background emission. 570~nm, which was used in sensitive imaging experiments, is nearly optimal in both cases, although 572~nm could have been used because the bulk sapphire background contribution was small in these experiments.