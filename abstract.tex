\begin{abstract}
The nEXO experiment is designed to search for zero-neutrino double beta decay of the isotope \textsuperscript{136}Xe, in order to better understand the nature of neutrinos. Since the daughter of this decay is barium (\textsuperscript{136}Ba), detecting the presence of \textsuperscript{136}Ba at a decay site (called "barium tagging") provides an additional discriminator to reject backgrounds in the search for this decay. This would involve detecting a single barium ion from within a macroscopic volume of liquid xenon. One proposed barium tagging method is to trap the barium ion in solid xenon (SXe) at the end of a cold probe, and then detect the ion by its fluorescence in the solid xenon. William M. Fairbank Jr.'s group at Colorado State University has been working toward this goal with steady success for some time.  In this thesis, I demonstrate successful detection of neutral Ba in SXe down to the single atom level, after deposition from vacuum.
\end{abstract}