\begin{abstract}
The nEXO experiment will search for neutrinoless double beta decay of the isotope \textsuperscript{136}Xe in a ton-scale liquid Xe time projection chamber, in order to probe the Majorana nature of neutrinos. Detecting the daughter (\textsuperscript{136}Ba) of double beta decay events, called barium tagging, is a technique under investigation which would provide a veto for a background-free measurement. This would involve detecting a single barium ion from within a macroscopic volume of liquid xenon. One proposed barium tagging method is to trap the barium ion in solid xenon (SXe) at the end of a cold probe, and then detect it by its fluorescence in the solid xenon.  In this thesis, new studies on the spectroscopy of deposits of Ba and Ba\textsuperscript{+} in solid Xe are presented.  Imaging of Ba atoms in solid Xe is demonstrated with ultimate sensitivity down to the single atom level.  Achievement of this level of sensitivity is a major step toward barium tagging by this method.

%in the search for this decay
\end{abstract}