\chapter{Conclusions}

Imaging of Ba atoms in SXe at the single-atom level has been demonstrated with deposits of small numbers of Ba\textsuperscript{+} ions in a focused laser region.  The 619-nm fluorescence peak, which has minimal bleaching, was used with around 0.2~mW of 570-nm excitation, over 10s of seconds of exposure.  This achievement is a major step toward the ability to tag Ba daughters in the nEXO neutrinoless double beta decay experiment.  Ba tagging is perhaps the only experimental method with the ability to probe the normal hierarchy of Majorana neutrino mass using neutrinoless double beta decay.

Studies of the spectroscopy of Ba and Ba\textsuperscript{+} in SXe are presented and have been published recently in \cite{Mong2015}.  Thermal history effects and temperature dependence of fluorescence have been discussed.  Bleaching of the fluorescence was studied at various excitation wavelengths and intensities, with comparison to a model of the rate equations.  These studies determined optimum conditions for single-atom level imaging with the 619-nm fluorescence, as well as in imaging the 577- and 591-nm fluorescence at the $\sim 10^{3}$-atom level.  Blue excitation of Ba\textsuperscript{+} deposits in SXe has revealed several newly reported emission peaks which are candidates for matrix-isolated Ba\textsuperscript{+} ions.

%neutralized Ba\textsuperscript{+} deposits in SXe have been presented.  Of particular interest was temperature and annealing effects, as well as bleaching effects of the 577-, 591-, and 619-nm fluorescence peaks.  Fluorescence was maximal in deposits made around 50~K and observed at 11~K.  The 577- and 591-nm peaks experience similarly rapid bleaching.  A model applied to bleaching data of the 591-nm peak demonstrated agreement with optical pumping rates of Ba in vacuum near the beginning of the bleaching process, with subsequent deviation.  The 619-nm peak experiences much lower bleaching rates.

%Understanding of these effects aided in the design of imaging experiments.  Imaging of Ba atoms in SXe in a focused laser region was demonstrated down to the $\leq 10^{3}$-atom level using the 577- and 591-nm peaks, and down to the single-atom level using the low-bleaching 619-nm peak.

\section{Future Work}
\label{sec:future}

The next step in the demonstration of Ba tagging in SXe will be imaging separate Ba atoms in SXe in images in which the laser beam is rastered across an area of the sample.  To achieve good scanned images, overcoming the surface background emission is the first priority.  Pre-bleaching of the sapphire window has been done in the first attempts at scanned images, though further study is needed to perfect the procedure.  Ba emission could be separated from background by different emission lifetimes, as is done in single molecule and quantum dot studies.  Fluorescence lifetimes of the 619-nm signal vs. the background using a < 1~ns pulsed laser and a photo-multiplier tube for fast detection could be explored, in order to determine whether a fluorescence time gating technique may remove background.  One exciting outcome of the atom counting inherent in scanned images is that they will provide the first measurement of the percentage of deposited Ba\textsuperscript{+} ions that neutralize in the SXe matrix and are visible as Ba atoms via the 619-nm fluorescence.

To improve Ba tagging efficiency, it would be good to be able to image Ba atoms in other matrix sites.  Exploration of re-pumping wavelengths for the 577- and 591-nm peaks may be done using tunable lasers in the infrared.  The ability to count atoms emitting with these different wavelengths would be very valuable.  It is not yet known if daughter Ba\textsuperscript{+} ions captured from LXe will remain as Ba\textsuperscript{+} or neutralize to Ba.  The ability to count Ba\textsuperscript{+} ions is also important and will be pursued by continuing study of the candidate Ba\textsuperscript{+} emission peaks with blue excitation.

Plans for ultimate demonstrations of Ba tagging in SXe include grabbing out of LXe on cold probes.  Construction of an apparatus is occurring simultaneously in our laboratory, in which a cold probe can be lowered into a LXe cell.  Initial tests of freezing LXe on Joule-Thompson probes have been successful.  Ba\textsuperscript{+} can be produced by laser ablation of Ba metal and drifted into the LXe, a process demonstrated in \cite{Kendy}, and onto the cold probe by electrodes.  In order to reach temperatures around 10~K, where the Ba fluorescence is strong, the probe can be raised to an isolated pumping region for further cooling and observation.

\vspace{10mm}

\begin{center}
\textbf{------}
\end{center}

The next breakthrough in particle/nuclear physics may lie in the discovery of Majorana neutrino mass, and the search for $0\nu\beta\beta$ is of key importance in this discovery.  Extreme sensitivity in experimental technique is required, and ingenious ideas, including the great challenge of Ba tagging must be invented and demonstrated to reduce competing backgrounds to zero over operating times of many years.  It is expected that unprecedented sensitivity in neutrinoless double beta decay will be reached in the second phase of nEXO, sufficient to cover the complete possibilities for the inverted neutrino mass hierarchy.  Based on progress that has been made, including this work, the future of Ba tagging is bright.
%pics?